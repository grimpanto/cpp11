\startenvironment env_book

% in LuaTeX and XeTeX, UTF-8 is on by default, thus not needed
\enableregime[utf-8]

% language mode: changes typesetting rules, quote signs etc.
\mainlanguage[de]  % \language[en] to change temporarily
\language[de]

% === Fonts ===

%\usetypescript[pagella]
%\setupbodyfont[libertine,12pt]
\setupbodyfont[pagella,12pt]
%\setupbodyfont[modern,12pt]

% ConTeXt's default \em is slanted, but italic is better
% but it makes no difference when using pagella!
\setupbodyfontenvironment[default][em=italic]


% === Modules ===

% http://wiki.contextgarden.net/Units, z.B.: 1.23 \Cubic \Meter \Per \Second
%\usemodule[units]

% Syntax highlighting
\usemodule[filter]
\usemodule[vim]

% Bibliography
\setupbibtex[database={bibliography},sort=author]
\setuppublications[alternative=apa]
\setupcite[left={[},right={]}]

% typesetting algorithms
%\usemodule[algorithmic]

% Annotations
\usemodule[annotation]

\define[2]\excerptCommand
    {\startblockquote%
    \quotation{\placeannotationcontent}\wordright{\tfx\placeannotationtitle}%
    \stopblockquote}

\defineannotation[excerpt]

\setupannotation[excerpt]
    [alternative=command,
    command=\excerptCommand]

\defineannotation[annotation]

\define[2]\annotationCommand
    {\textrule{\placeannotationtitle}\placeannotationcontent\blank[small]\textrule}

\setupannotation[alternative=command,
    command=\annotationCommand,
    spacebefore=none,
    before={}]

\setupbackground[
    backgroundcolor=veryverylightgray,
    backgroundoffset=1mm,width=local,align=middle,%
       top={\blank[big]},bottom={\blank[big]}%,backgroundcorner=rectangular,
    %      %backgroundradius=1mm]
    ]

\definestartstop
    [innerdrawer]
    [%commands={\background},
     before={\background},
     after={}]

\definestartstop
    [drawer]
    [before={\blank[2*big]\startinnerdrawer\startnarrower},
     after={\stopnarrower\stopinnerdrawer\blank[2*big]}]

%\definestartstop
%    [drawer]
%    [before={\blank\startbackground[%backgroundoffset=3mm,
%      backgroundcolor=veryverylightgray%,backgroundcorner=rectangular,
%      %backgroundradius=1mm]},
%      ]},
%     after={\stopbackground\blank}]

%\definetextbackground
%    [drawer]
%    [ location=text,
%      background=color,
%      backgroundcolor=veryverylightgray,
%      frame=off,
%      corner=round,
%      radius=0.8\lineheight,
%    ]

%\defineannotation[drawer]

%\define[2]\drawerCommand{
%    \blank\startbackground[backgroundcolor=veryverylightgray,backgroundcorner=rectangular,backgroundradius=1mm]\startframedtext[width=local,frame=off,framecorner=round,frameradius=2mm]\placeannotationcontent\stopframedtext\stopbackground\blank}

%\define[2]\drawerCommand{
%    \blank\startbackground[backgroundoffset=3mm,backgroundcolor=veryverylightgray,backgroundcorner=rectangular,backgroundradius=1mm]\placeannotationcontent\stopbackground\blank}


%\setupannotation[drawer]
%    [alternative=command,
%    command=\drawerCommand]%,
%%    text=Drawer]

% Tikz
%\usemodule[tikz]
%\usetikzlibrary[arrows,decorations.pathmorphing,backgrounds,positioning,fit,petri]
%\usetikzlibrary[graphs]
%\usetikzlibrary[trees]
%%\usetikzlibrary[datavisualization]
%\usetikzlibrary[graphdrawing]
%\usegdlibrary[trees,layered,force]

% useful symbols
\usesymbols[mis]
\setupsymbolset[text]

\usesymbols[mvs]
\setupsymbolset[europe]
\setupsymbolset[martinvogel 2]

% can be used in the following way

% eur
%\euro
%{\ss\bf \euro}
%{\tt \euro}
%
%% mis
%\symbol[bullet]
%\symbol[dash]
%\symbol[star]
%\symbol[triangle]
%\symbol[circle]
%\symbol[square]
%\symbol[diamond]
%
%\symbol[checkmark]
%
%\symbol[S] % section mark like a paragraph sign
%
%% mvs
%\symbol[EUR]
%\symbol[EURhv]
%\symbol[EURcr]
%\symbol[EURtm]
%
%\symbol[PointingHand]
%\symbol[WritingHand]
%\symbol[CrossedBox]
%
%\symbol[Lightning]
%\symbol[Coffeecup]
%\symbol[YinYang]
%
%\symbol[Info]
%\symbol[Stopsign]
%\symbol[Radioactivity]
%
%\symbol[Letter]
%\symbol[Telephone]
%\symbol[ClockLogo]
%\symbol[Printer]
%\symbol[Keyboard]
%
%\symbol[Conclusion]
%\symbol[Equivalence]
%\symbol[Congruent]
%\symbol[NotCongruent]
%
%\symbol[Frowny]
%\symbol[Smiley]
%
%\symbol[CutRight]
%\symbol[CutLeft]
%\symbol[LeftScissors]
%\symbol[RightScissors]

% Step charts
%\usemodule[steps]

% === Colors ===

\setupcolors[state=start]  % otherwise grey
%\definecolor[headingcolor][r=1,b=0.4]

\definecolor[verylightgray][1.05(lightgray)]
\definecolor[veryverylightgray][1.08(lightgray)]
\definecolor[typecolor][0.65(blue)]
\definecolor[filecolor][0.4(blue)]
\definecolor[linkcolor][0.85(blue)]
\definecolor[warncolor][0.85(red)]


% === Interaction ===

% Enable hyperlinks
\setupinteraction[state=start,color=middleblue]

\setupurl[color=linkcolor,style=\tf]


% === Headers ===
\setuphead[chapter,section,subsection,subsubsection][numberstyle=bold]
\setuphead
    [chapter,title]
    [style={\ss\bfc},
     color=headingcolor]
\setuphead
    [chapter]
    [numberstyle=bold,
     before=\hairline\blank,
     after=\nowhitespace\hairline\blank,
     page=yes,continue=no]
\setuphead[section,subject][style={\ss\bfb}]
\setuphead[subsection,subsubject][style={\ss\bfa},before={\blank}]
\setuphead[subsubsection,subsubsubject][style={\ss\bf}]
\setuphead[subsubsubject][before={}]

% === Misc ===

\definedescription
    [description]

\setupdescriptions
    [description]
    [alternative=hanging,headstyle=bold,width=fit]

\definedescription
    [concept]
    [alternative=serried,headstyle=bold,width=broad]

\defineframedtext
  [framedcode]
  [%strut=yes,
   %offset=13mm,
   %width=\textwidth,
   width=local,
   frame=off,
   %margin=yes,
   %background=color,
   backgroundcolor=verylightgray,
   align=right]

% typesetting code without syntax highlighting
\definetyping[code][numbering=line,
    %option=c,
    before={\startframedcode},
    after={\stopframedcode}]

\definetyping[minicode][numbering=line,
    %option=c,
    before={\startframedcode\tfx},
    after={\stopframedcode}]

\setuptype[color=typecolor] %, style=bold]

\definedescription[latexdesc][
    headstyle=bold, style=normal, align=left, location=hanging,
    width=broad, margin=1cm]

% used for typesetting file names
\define[1]\file{\color[filecolor]{\tt #1}}

% maybe it is better to use the module annotation
%\defineframedtext
%    [note]
%    [indenting=never,
%     frame=on,
%     width=local,
%     offset=5px]

\definesynonyms[definition][definitions][\infull]
\setupsynonyms[definition][criterium=all]

% === Highlighting ===

\def\codingtitle#1{#1}
\definefloat[coding]
\setupcaption[coding][location=bottom,number=no,align=inner,command=\codingtitle]

%\setupvimtyping[%alternative=blackandwhite,
%    numbering=yes, % vimtyping
%    numberdistance=2em, % vimtyping: distance between number and code
%    %strip=yes, % vimtyping: strip leading spaces
%    before={\blank[big]\startbackground[backgroundcolor=veryverylightgray,%
%      backgroundcorner=round,backgroundradius=2mm,backgroundoffset=0mm,%
%      top={},bottom={}]
%      \startframedtext[width=local,frame=off]},
%    after={\stopframedtext\stopbackground\blank[big]}]

\setupvimtyping[%alternative=blackandwhite,
    numbering=yes,% vimtyping
    numberdistance=1em, % vimtyping: distance between number and code
    strip=yes, % vimtyping: strip leading spaces
    before={\blank[big]%
      \startframedtext[width=local,frame=on,frameradius=2mm,framecorner=round]},
    after={\stopframedtext\blank[big]}]


%\setupbackground[backgroundoffset=5mm]

%\setupvimtyping[%alternative=blackandwhite,
%    numbering=yes, % vimtyping
%    numberdistance=2em, % vimtyping: distance between number and code
%    %strip=yes, % vimtyping: strip leading spaces
%    before={\blank\startbackground[%backgroundoffset=3mm,
%      backgroundcolor=veryverylightgray,backgroundcorner=round,backgroundradius=2mm]},
%    after={\stopbackground\blank}]

\definevimtyping[python][syntax=python]
\definevimtyping[cpp][syntax=cpp]
\definevimtyping[java][syntax=java]
\definevimtyping[sh][]

% === Figures ===
\setupexternalfigures[location={local,default}]


% === Typography ===
\setupwhitespace[medium]

% punctation into the margin: http://wiki.contextgarden.net/Protrusion
\definefontfeature
  [default]
  [default]
  [protrusion=quality,expansion=quality]
\setupalign[hz,hanging]

% set inter-paragraph spacing
\setupwhitespace[medium]

% to indent paragraphs
%\setupindenting[medium, yes]

% to be tolerant for vertical and horizontal typesetting
% possible values: verystrict (default), strict, tolerant, verytolerant, stretch
\setuptolerance[horizontal,tolerant]
%\setuptolerance[vertical,tolerant]

% === Layout ===
\startmode[screen]
\setuppapersize[S6][S6]
\setuplayout[width=fit,
    height=fit,
    rightmargin=15mm,
    leftmargin=15mm,
    %textwidth=15cm,
    textdistance=0mm,
    textmargin=10mm,
    clipoffset=0mm,
    leftedgedistance=0mm,
    leftedge=0mm,
    %location=left,
    horoffset=0mm,
    leftmargindistance=0mm,
    rightmargindistance=0mm,
    %topspace=1.8cm,
    %header=0pt,
    topspace=1.0cm,
    header=1.0cm,
    footer=1.2cm,
    bottomspace=0.5cm,
    backspace=2cm,
    location=singlesided]
\stopmode

\startmode[print]
\setuppapersize[A4][A4]
\stopmode

% table of contents

\unprotect
\definelistalternative
  [dotfix]
  [distance=0pt,
   width=2em,
   stretch=10em,
   filler=\hskip.5em\gleaders\hbox to .5em{\hss.\hss}\hfill\relax,
   renderingsetup=\??listrenderings:abc]
 \protect

 \setupcombinedlist
  [section]
  [alternative=dotfix]

\setupcombinedlist
  [subsection]
  [alternative=dotfix]

%\setuplist[interaction=all]

\setuplist[chapter]
  [alternative=b,
   style=sansbold,
   before={\blank[4*big]}]

\setuplist
  [section]
  [width=2em]

\setuplist
  [subsection]
  [width=2.7em,margin=2em]

%\setupwhitespace
%  [medium]

%\startsectionblockenvironment[frontpart]
%  \setupinterlinespace [line=1.5ex]
%\stopsectionblockenvironment

\stopenvironment
