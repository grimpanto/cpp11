
\startcomponent comp_preface
\product prod_book

\chapter{Vorwort}
Die Idee zu diesem Buch kam mir, da ich kein Buch gefunden habe, das
einem Programmierer bei seinem Einstieg in die produktive Programmierung
mit \cppXI
unterstützt.

Das erste Mal habe ich mich ca. vor 25 Jahren mit \cpp beschäftigt und
etliche Jahre mit dieser Programmiersprache entwickelt. Dann wurden
meine Hauptwerkzeuge für die Entwicklung hauptsächlich Java und Python.
Nach einer langen Periode in der Entwicklung von verteilten Systemen bin
ich vor einiger Zeit wieder mit \cpp und im speziellen mit \cppXI
in Berührung gekommen. Und da hat sich {\em einiges} geändert.

Wie beginnt man am Besten mit einer Sprache, die man schon kennt, aber
die sich stark verändert hat. \cppXI fühlt sich im Vergleich zu \cpp aus
den 80er Jahren wie eine {\em neue} Programmiersprache an. Also stand
ich da als Programmierer mit 30 Programmiererfahrung und hatte eine neue
Programmiersprache vor mir.

Also habe ich jede Menge Bücher gekauft, durchgearbeitet und sehr viel
programmiert. Wäre ich mit entsprechenden Unterlagen nicht schneller
produktiv gewesen? Ja, ich denke schon. In diesem Sinne hoffe ich, dass
dieses Buch meine Leser schnell an Ihr Ziel bringt, \cpp produktiv zu
verwenden!

Neunkirchen, im September 2014\crlf
Günter Kolousek

\stopcomponent
