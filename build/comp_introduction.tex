
\startcomponent comp_introduction
\product prod_book

\chapter{Einführung in \cpp~}
\section[charakterisierung-von]{Charakterisierung von \cpp}

Bei \cpp handelt es sich um eine objektorientierte Programmiersprache,
die eine der weltweit am häufigsten eingesetzten Programmiersprachen
ist.

Die Entwicklung der Programmiersprache \cpp wurde von Bjarne Stroustrup
1979 als eine objektorientierte Erweiterung zur Programmiersprache C
unter dem Namen \quotation{C with Classes} begonnen. 1984 erfolgte die
Umbenennung in \cpp und der erste kommerzielle Compiler erschien 1985.
Seitdem entwickelt sich \cpp permanent weiter und es folgten im Jahr
1989 und 2003 die Versionen \type{C++98} und \type{C++03}. Diese
Entwicklung hatte im Jahr 2011 mit \cppXI einen vorläufigen Höhepunkt
mit vielen substanziellen Neuerungen. \cppXI legte den Grundstein für
die moderne Entwicklung mit \cpp!

2014 wurde eine Zwischenversion mit Berichtigungen und kleinen
Erweiterungen herausgebracht, die als \cppXIV bekannt ist. Man sieht,
dass \cpp sowohl eine lange Vergangenheit hat und es permanent weiter
entwickelt wird: 2017 soll die nächste große Version von
\cpp erscheinen.

In diesem Sinne spreche ich in diesem Buch von \quotation{\cpp} wenn ich
eine beliebige Version von \cpp meine, von \quotation{\cppXI} wenn ich
mich auf \cppXI oder \cppXIV
beziehe und von \quotation{\cppXIV} wenn ich speziell \cppXIV anspreche.

Der Erfinder von \cpp, Bjarne Stroustrup, beschreibt \cpp als eine
allgemein verwendbare Programmiersprache zum Entwickeln und Verwenden
von eleganten und effizienten Abstraktionen.

Bjarne Stroustrup hat zwei grundlegende Entwurfsprinzipien für \cpp
festgelegt: Einerseits soll es möglich sein, mit \cpp nahe der Hardware
zu programmieren und andererseits sollen \cpp Programme hoch performant
sein.

Aus diesen beiden Vorgaben ergibt sich, dass \cpp eine überragende
Bedeutung im Bereich der Systemprogrammierung hat. Mit \cpp werden
Treiberprogramme, ganze Betriebssysteme, Compiler, Software für
eingebettete Systeme (engl. embedded systems) wie Software in Autos,
Industrieanlagen, Handys,\ldots{} entwickelt.

\cpp wird nicht nur zur Systemprogrammierung, sondern auch großteils zur
Anwendungsprogrammierung verwendet. Die Arten dieser Anwendungen sind
sehr vielfältig. Es werden Systeme für die Finanzmärkte, Simulationen,
Büroanwendungen, Grafikprogramme, Spiele,\ldots{} entwickelt. Große
Bedeutung hat \cpp auch im Grafikbereich, in der Verarbeitung von
Multimediadaten wie Bildern, Video und Audioinformationen, bei der
Visualisierung, in der Meteorologie, Physik, Genetik,\ldots{}

Auch die Größe der Computer ist unterschiedlich auf denen diese
Programmiersprache zum Einsatz kommt. Abgesehen von den eingebetteten
Systemen, findet sich \cpp im Einsatz auf Personal Computern, in
Serverlösungen bis hin zu Supercomputer.

Große Firmen wie Adobe Systems Inc.~(z.B.~Photoshop), Apple Inc. (z.B.
Teile von Mac OSX), Google Inc. (z.B.~Google Chrome), Microsoft
Corporation (z.B.~Teile von Microsoft Windows), Mozilla Corporation
(z.B.~Firefox), Oracle Corporation (z.B.~MySQL) und viele mehr setzen
\cpp in ihren Produkten ein. Bei den hier genannten Firmen handelt es
sich um eine subjektive Auswahl ausgewählt nach bestem Wissen und
Gewissen.

\cpp wird allerdings von keiner Firma kontrolliert, sondern von einem
internationalen Komitee entwickelt. Die einzelnen Versionen von \cpp
werden als Standard bei der ISO (International Organization for
Standardization) herausgebracht und haben normativen Charakter.

Mit der Version von \cpp aus dem Jahre 2011, also \cppXI, ist ein
weiterer wichtiger Aspekt hinzugekommen: \cpp soll leicht erlernbar
sein. Um mit {\em allen} Aspekten von \cpp vertraut zu sein, benötigt
man viele Jahre intensives Auseinandersetzen und Übung. Um produktiv
Programme in \cppXI zu entwickeln, benötigt man nur {\em wenige Monate}.

Und genau das ist das Ziel dieses Buches: \cpp in dem Maße zu erlernen,
dass es möglichst schnell erlernt und produktiv verwendet werden kann!

\section[merkmale-von]{Merkmale von \cpp}

\subsubject[typisierung]{Typisierung}

Je nach Art wie Variablen einem Typ zugeordnet wird, unterscheidet man
verschiedene Arten der Typisierung. \cpp wird als {\em statisch} und
{\em streng} getypte Programmiersprache kategorisiert.

Statische Typisierung bedeutet, dass der Typ einer Variable schon bei
Kompilierung feststeht. Kompilieren bedeutet, dass der Quelltext in eine
ausführbare Datei übersetzt wird. Bei dynamisch getypten
Programmiersprachen ist solch eine Kompilierung in der Regel nicht
notwendig, da der Typ einer Variable erst zur Laufzeit feststeht.

Strenge Typisierung bedeutet hingegen, dass Operationen nur auf
kompatiblen Typen durchgeführt werden und Typkonversionen nur beschränkt
implizit durchgeführt werden. Das Gegenstück ist die schwache
Typisierung, die Operationen weitgehend zwischen beliebigen Typen
ermöglicht.

Es gibt keine exakte Definition von strenger oder schwacher Typisierung.

\subsubject[unterstützte-programmierparadigmen]{Unterstützte
Programmierparadigmen}

\cpp ist eine Programmiersprache, die mehrere Programmierparadigmen
unterstützt:

\startdescription{Imperative und prozedurale Programmierung}
  Da du ja schon Programmierer in einer objektorientierten
  Programmiersprache bist, bist du prinzipiell auch in der Lage
  imperativ (basierend auf Befehlen und Anweisungen) und prozedural
  (strukturiert mittels Prozeduren und Funktionen) zu programmieren.
\stopdescription

\startdescription{Generische Programmierung}
  Mittels der generischen Programmierung versucht man Algorithmen
  möglichst allgemein zu definieren, sodass diese später mittels
  verschiedensten Datentypen verwendet werden können. Der Vorteil, der
  gegenüber einer überhandnehmenden Objektorientierung gesehen wird,
  ist, dass es in vielen Fällen eine einfachere und performantere
  Implementierung in statisch getypten Programmiersprachen bietet.

  \cpp bietet hierfür mittels der Templates einen ausgefeilten
  Mechanismus an, der generische Programmierung ermöglicht.
\stopdescription

\startdescription{Objektorientierte Programmierung}
  Die objekt-orientierte Programmierung ist eines der wichtigsten
  Paradigmen und bietet richtig angewandt enorme Vorteile bei der
  Abbildung des Problemraumes in ein adäquates Modell.

  In \cpp nimmt die objektorientierte Programmierung nicht mehr einen so
  großen Stellenwert ein wie früher. Das liegt daran, dass die
  generische Programmierung einen großen Anteil übernommen hat.
\stopdescription

\startdescription{Funktionale Programmierung}
  In der funktionalen Programmierung bestehen Programme ausschließlich
  aus Funktionen, deren Verarbeitung wie die Berechnung mathematischer
  Funktionen gehandhabt wird. Diese Funktionen haben keinen Zustand und
  auch keine Nebeneffekte. Das kann Vorteile bringen und kann unter
  \cpp auch in gewissen Maßen umgesetzt und genutzt werden.

  Auch wenn \cpp keine rein funktionale Sprache ist, bietet es doch
  mittels den Sprachkonstrukten Funktionen, Funktionsobjekten,
  Lambda-Ausdrücken und der Standardbibliothek (\type{std::function},
  \type{std::bind}, Higher-order Funktionen in den Algorithmen,
  \type{std::async}) Möglichkeiten, um funktional zu programmieren.
\stopdescription

Im weiteren Verlauf dieses Buches werden wir diese verschiedenen Seiten
von \cpp noch kennenlernen.

\section[benötigte-software]{Benötigte Software}

In diesem Abschnitt werde ich die {\em prinzipiellen} Möglichkeiten
beschreiben, mit der \cpp Programme erstellt werden können.

Es gibt mehrere Möglichkeiten, wie du deine Werkzeuge auswählen kannst.

\subsubject[entwicklungswerkzeuge]{Entwicklungswerkzeuge}

Je nach Betriebssystem gibt es für die Entwicklungswerkzeuge wie
Compiler, Linker und Debugger verschiedene Möglichkeiten, die im
Abschnitt \in[devtools]
auf der Seite \at[devtools] prinzipiell besprochen werden. Es liegt an
dir eine der Varianten zu wählen.

\subsubject[entwicklungsumgebung]{Entwicklungsumgebung}

Zum Erstellen und Verwalten des \cpp Sourcecodes wird entweder eine
integrierte Entwicklungsumgebung (IDE, engl. integrated develeopment
environment) oder ein vernünftiger Editor benötigt.

Auch hier gibt es je nach verwendetem Betriebssystem mehrere
Wahlmöglichkeiten. Es gibt allerdings auch die ausgezeichnete
Entwicklungsumgebung QtCreator, die plattformübergreifend zur Verfügung
steht.

\subsubject[framework]{Framework}

Um Programme mit grafischer Oberfläche, mit Netzwerkfunktionalitäten
oder mit Datenbankzugriffen zu entwickeln, wird noch ein Framework
benötigt.

Bezüglich des verwendeten Frameworks ist die Zielplattform von
entscheidender Bedeutung. Will man seine Programme exakt für Windows auf
den Markt bringen, dann wählt man sinnvollerweise die entsprechende
Möglichkeit von Microsoft. Ähnlich sieht die Situation für die Plattform
Mac OSX und Linux aus.

Ist es allerdings das Ziel seine Programme sowohl unter Windows als auch
unter Mac OSX und Linux zur Verfügung stellen zu wollen, dann bietet
sich das plattformübergreifende Framework Qt an. Dieses Framework bietet
die folgenden Vorteile:

\startitemize[packed]
\item
  Da es plattformübergreifend ist, steht es für die Betriebssysteme
  Windows, Mac OSX und Linux zur Verfügung. Außerdem kann man damit auch
  Programme für die Betriebssysteme Android und iOS entwickeln!
\item
  Es bietet uns die Möglichkeit portablen Code für grafische
  Oberflächen, Netzwerkfunktionen und Datenbankzugriffen produktiv zu
  entwickeln.
\item
  Mit Qt wird außerdem die Entwicklungsumgebung QtCreator mitgeliefert,
  die für die Entwicklung von \cpp Programmen ausgezeichnet geeignet
  ist.
\item
  Für die Entwicklung von privaten Programmen oder von Open Source
  Programmen ist die Verwendung von Qt kostenlos.
\stopitemize

Bei diesen überwältigenden Vorteilen stellt sich natürlich auch die
Frage nach etwaigen Nachteilen:

\startitemize
\item
  Unter Umständen ist dieses Framework nicht für {\em jeden}
  Anwendungsfall und für {\em jeden} Entwickler die optimale Lösung.
\item
  Für die Entwicklung von kommerzieller Software muss eine Lizenz
  gekauft werden, wogegen die Entwicklung von privaten Programmen und
  Open Source Programmen kostenlos möglich ist.

  \startannotation{Bemerkung}Eigentlich ist es so, dass sogar die
  Entwicklung von proprietärer Software unter Einhaltung der LGPL
  (Lesser General Public License) möglich ist. Dies kann dadurch
  erreicht werden, dass das infrage kommende Programm dynamisch gegen
  die Qt Bibliotheken gelinkt wird.\stopannotation
\stopitemize

Das SDK (software development kit) von Qt kann von
\useURL[url1][http://qt-project.org]\from[url1] heruntergeladen und
installiert werden.

Da du ein Programmierer bist, wird in weiteren davon ausgegangen, dass
du dir deine Entwicklungsumgebung selbständig aussuchst und
installierst.

\section[ausführbares]{Ausführbares \cpp{}}

\cpp ist eine Programmiersprache bei der Quelltext zuerst in die
Maschinensprache der jeweiligen Zielplattform übersetzt werden muss. Das
wird mittels eines Übersetzungsvorganges (auch Kompilierung genannt) von
einem Compiler (eingedeutscht von engl. {\em compiler}) durchgeführt,
der den Quelltext (engl.~{\em source code}) in eine ausführbare Form
(engl. {\em executable}) übersetzt.

Besteht das Programm aus mehreren Programmdateien, die getrennt
übersetzt worden sind, dann müssen diese noch miteinander verbunden
werden. Dieses Binden wird durch einen Linker (eingedeutscht von engl.
{\em linker} oder {\em link editor}) durchgeführt. In der Regel muss man
sich darum nicht kümmern, da dies durch das Compiler-Frontend, also das
Compiler-Programm selbsttätig vorgenommen wird.

Die gängigen Implementierungen von \cpp erwarten sich, dass der
Quelltext in Dateien abgespeichert ist. Als Dateierweiterung (engl.
{\em filename extension})
\index{Dateierweiterung}\index{Dateierweiterung+\cpp Datei} wird in
\cpp meist \type{.cpp} verwendet, aber auch \type{.C}, \type{.cc},
\type{.cxx} oder \type{.c++} kommen zum Einsatz. Wir verwenden
\type{.cpp}.

Weiters gibt es sogenannte Headerdateien (wird später noch erklärt),
deren Dateierweiterung meist \type{.h} besitzt, aber auch \type{.H} oder
\type{.hpp} kommen zeitweise zum Einsatz. Wir verwenden \type{.h}.

Im Abschnitt \in[compilation] auf Seite \at[compilation] findest du eine
Kurzanleitung wie aus einem \cpp Quelltext mittels einer Shell eine
ausführbare Datei erstellt werden kann.

Danach kann das ausführbare Programm gestartet werden, es entsteht ein
Prozess, der sich irgendwann einmal beenden wird. Beim Beenden gibt der
Prozess einen Exitcode an den Prozess zurück, der ihn gestartet hat.
Mittels dieses Exitcodes kann der beendete Prozess eine Information an
den aufrufenden Prozess zurückgeben.

Im Anhang \in[access_exit_code] auf der Seite
\at[access_exit_code] befindet sich eine Kurzanleitung wie du je,
verwendetem Betriebssystem, auf diesen Exitcode mittels der Shell
zugreifen kannst.

\stopcomponent
