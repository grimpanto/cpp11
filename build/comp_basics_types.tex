
\startcomponent comp_basics_types
\product prod_book

\chapter{Grundlagen zu den Datentypen}
\startdrawer

In diesem Kapitel erhältst du einen Überblick über die eingebauten
Datentypen von \cpp.

Hier werden wichtige Grundlagen erklärt und mit Beispielen untermauert,
sodass die weiteren Datentypen eingeordnet und verstanden werden können.
\stopdrawer

\section[einteilung-der-datentypen]{Einteilung der Datentypen}

Die Datentypen von \cpp können in eingebaute Datentypen (engl.
{\em built-in}) und benutzerdefinierte Datentypen (engl.
{\em user-defined}) eingeteilt werden.

Die eingebauten Typen unterteilen sich wiederum in die fundamentalen
Datentypen und in Typen, die mittels Deklaratoroperatoren (engl.
{\em declarator operator}) aus den fundamentalen Datentypen abgeleitet
(erzeugt) werden können.

Die fundamentalen Datentypen sind:

\startitemize[packed]
\item
  Boolescher Typ (engl. {\em boolean type}) für Wahrheitswerte:
  \type{bool}.
\item
  Zeichentyp (engl. {\em character type}): \type{char}, \type{wchar_t},
  \type{char16_t} und \type{char32_t}.
\item
  Ganzzahltyp (engl. {\em integer type}): \type{short}, \type{int},
  \type{long}, \type{long long}. Von nun an verwende ich der Einfachheit
  halber einfach den eingedeutschten Begriff Integer anstatt
  Ganzzahltyp.
\item
  Gleitkommazahltyp (engl. {\em floating-point type}): \type{float},
  \type{double}, \type{long double}.
\item
  Der Typ \type{void}, der angibt, dass keine Information über den Typ
  vorhanden ist.
\stopitemize

Aus diesen fundamentalen Datentypen und auch den benutzerdefinierten
Datentypen können mittels der folgenden Deklaratoroperatoren neue
Datentypen definiert werden:

\startitemize
\item
  Zeigertyp (engl. {\em pointer type}), wie z.B. \type{int*}, der einen
  Zeiger (eingedeutscht {\em Pointer}) auf einen Integer darstellt. Ein
  Pointer stellt eine Adresse auf einen Speicherbereich dar. In dem
  konkreten Beispiel ist es eben die Adresse auf einen Speicherbereich,
  der als Integer vom Typ \type{int} betrachtet wird.

  Pointer sind ein wichtiges Konzept in \cpp, aber die Handhabung ist
  nicht immer ganz einfach. Wir werden die notwendigen Grundlagen nach
  und nach durcharbeiten.
\item
  Arraytyp (engl. {\em array type}), wie z.B. \type{char[10]}, das ein
  Array von Zeichen darstellt. Ein Array ist ein Speicherbereich fester
  Größe, der eine Folge von Werten eines bestimmten Typs beinhaltet. Bei
  \type{char[10]} handelt es sich um einen Speicherbereich, der genau 10
  Zeichen vom Typ \type{char} enthält.
\item
  Referenztyp (engl. {\em reference type}), wie z.B. \type{double&}, das
  eine Referenz auf einen Speicherbereich vom Typ \type{double} ist.
  Eine Referenz ist nichts anderes als ein anderer Name für eine
  bestimmte Speicherstelle. Mehr dazu später.
\stopitemize

Es gibt noch ein paar wichtige Begriffe:

\startitemize
\item
  Der boolesche Typ, die Zeichentypen und Integer-Typen werden als
  integrale Typen (engl. {\em integral type}) bezeichnet. Ein integraler
  Typ ist dadurch gekennzeichnet, dass man mit diesem rechnen kann und
  bitweise logische Operationen (z.B. das bitweise Oder) durchführen
  kann!
\item
  Ein arithmetischer Typ ist entweder ein integraler Typ oder ein
  Gleitkommazahltyp. Logischerweise kann man mit einem arithmetischen
  Typ arithmetische Operationen durchführen. Im Gegensatz zu integralen
  Typen kann man Gleitkommazahlen allerdings nicht mit bitweisen
  logischen Operationen verwenden.

  Die unterschiedlichen arithmetischen Datentypen können in Zuweisungen
  und in Ausdrücken (wie z.B. Berechnungen) beliebig miteinander
  verknüpft werden. Dabei werden implizite Konvertierungen vorgenommen.
  Mehr dazu später.
\stopitemize

Zusätzlich zu den eingebauten Datentypen können noch gänzlich neue
Typen, sogenannte benutzerdefinierte Datentypen, definiert werden. Dabei
handelt es sich um Aufzählungen (engl. {\em enumeration}) und Klassen
(engl. {\em class}). Siehe später.

\section[implementationaspects]{Implementierungsspezifische Aspekte}

Wie schon erwähnt, gibt es etliche Teile von \cpp, die von der konkreten
Implementierung abhängig sind, da sie nicht spezifiziert sind. Das hat
seinen Grund darin, dass jede Implementierung auf ein spezielles System
zugeschnitten werden kann. Eine der Zielsetzungen von \cpp ist es,
Hardware-nahe Programmierung durchführen zu können. Das ist speziell in
der Programmierung eingebetteter Systeme wichtig, da zum Beispiel bei
der Programmierung eines Handys nicht der gleiche Prozessor
vorausgesetzt werden kann wie bei der Programmierung einer
Desktop-Anwendung.

So verhält es sich auch mit den Datentypen, deren exakte Größe nicht
spezifiziert ist, sondern nur die Verhältnisse der Größen zueinander. So
kann man sich nicht darauf verlassen, dass eine ganze Zahl aus einer
bestimmten Anzahl von Bytes zusammengesetzt ist. Damit ist die größte
und auch die kleinste darstellbare Zahl des Typs \type{int} nicht
definiert! Es ist lediglich festgeschrieben, dass die Größe eines
\type{int} mindestens so groß sein muss wie die Größe eines \type{char}.

Schreibe folgenden Quelltext in ein Programm \type{size.cpp}:

\startcpp
#include <iostream>

using namespace std;

int main() {
    cout << sizeof(char) << endl;
    cout << sizeof(short) << endl;
    cout << sizeof(int) << endl;
    cout << sizeof(long long) << endl;
}
\stopcpp

Bei mir unter Linux in einer 32 Bit Variante kommt es zu folgenden
Ausgaben:

\startsh
1
2
4
8
\stopsh

Wichtig zu wissen ist, dass in \cpp die Speichergrößen in Vielfachen der
Größe eines \type{char} ausgedrückt werden. Damit ist per Definition
\type{sizeof(char)} immer 1! Damit ist auf meinem System ein \type{int}
vier Mal so groß wie ein \type{char}. Wie viel Bits ein \type{char}
wirklich hat ist damit nicht zu erfahren. Meist hat ein \type{char} 8
Bits, also ein Oktett. Üblicherweise hat ein Byte die Größe eines
Oktetts. Ich gehe in weiterer Folge davon aus, dass ein \type{char} die
Größe eines Bytes aufweist.

In diesem Zusammenhang werden wir auf das Schlüsselwort
\type{static_assert} kennenlernen, das vom Compiler selber verwendet und
von diesem ausgewertet wird. Hänge die folgende Zeile an das Programm
an:

\startcpp
static_assert(3 == sizeof(int),
              "Größe eines ints hat nicht 3 Bytes");
\stopcpp

Beim Versuch das Programm jetzt zu übersetzen, wird der {\em Compiler}
eine Fehlermeldung liefern, da ein \type{int} mit an Sicherheit
grenzender Wahrscheinlichkeit auf keinem System genau 3 Bytes lang ist
oder genauer gesagt die dreifache Größe eines \type{char} aufweist. Dazu
muss der Compiler zur Übersetzungszeit diese Anweisung ausführen. Dazu
wertet der Compiler die Bedingung aus und wenn diese nicht erfüllt ist,
dann wird letztendlich der Übersetzungsvorgang mit einer Fehlermeldung
abgebrochen.

Ersetzt du diese Zeile durch die folgende Zeile, dann sollte auf einem
PC mit Windows, Linux oder einem MacOSX der Compiler dies übersetzen:

\startcpp
static_assert(4 <= sizeof(int),
              "int hat nicht mindestens 4 Bytes");
\stopcpp

In den folgenden Abschnitten werden wir etwas genauer auf diese
Problematik eingehen, werden aber nicht alle Regeln angeben, da dies
einerseits in dieser Form nicht möglich ist und auch nicht den
Zielsetzungen dieses Buches entspricht.

\section[identifier]{Bezeichner}

Wie schon besprochen muss jeder Bezeichner (engl. {\em identifier}, auch
Name genannt) in einem \cpp Programm vor der Verwendung deklariert
werden. Damit wird dem Compiler einerseits ein Name und andererseits ein
zugeordneter Typ bekannt gemacht.

Die folgenden Regeln gelten für den Aufbau von Bezeichnern:

\startitemize
\item
  Bezeichner bestehen in \cpp aus Buchstaben, Ziffern und Unterstrichen
  (\type{_}), wobei ein Bezeichner nicht mit einer Ziffer beginnen darf.
  Groß- und Kleinbuchstaben werden unterschieden.
\item
  Allerdings dürfen Bezeichner nicht mit einem eingebauten Schlüsselwort
  übereinstimmen. Da es mehr als 80 Schlüsselwörter in \cpp gibt, nehme
  ich davon Abstand diese hier anzuführen. Ein Versuch ein Keyword als
  Bezeichner zu verwenden wird der Compiler zuverlässig melden.
\item
  Beginne einen Bezeichner nicht mit einem oder mehreren Unterstrichen,
  wie z.B. \type{_tmp}!

  Die genaue Regel ist komplizierter und nachfolgend angegeben: Nicht
  lokale Bezeichner dürfen nicht mit genau einem Unterstrich beginnen
  (z.B. \type{_error}), da diese für die Implementierung beziehungsweise
  das Laufzeitsystem reserviert sind. Generell dürfen Bezeichner nicht
  mit zwei Unterstrichen (z.B. \type{__status}) oder genau einem
  Unterstrich gefolgt von einem Großbuchstaben (z.B. \type{_State})
  beginnen, da auch diese reserviert sind.
\stopitemize

Es gibt viele verschiedene Arten, wie man seine Bezeichner aufbauen
kann. Ich verwende die folgende Art:

\startitemize
\item
  Handelt es sich um einen benutzerdefinierten Typ, dann beginnt dieser
  mit einem Großbuchstaben. Setzt sich dieser Typ aus mehreren
  Teilwörtern zusammen, dann wird jedes Teilwort ebenfalls groß
  geschrieben. Ein Beispiel: \type{NamesDirectory}.

  Eingebaute Datentypen und Datentypen aus der Standardbibliothek von
  \cpp
  beginnen mit Kleinbuchstaben. Damit ist eine klare Unterscheidung
  möglich.
\item
  Handelt es sich um eine Variable oder eine Funktion, dann beginnen
  diese mit einem Kleinbuchstaben und die etwaigen Teilwörter sind
  jeweils durch einen Unterstrich (engl. {\em underscore}) voneinander
  getrennt. Ein Beispiel: \type{first_name}.

  Methoden (engl. {\em method}) werden in \cpp ebenfalls als Funktionen
  (engl. {\em function}) aufgefasst. Will man sich auf Methoden
  beschränken, dann verwendet man den Begriff Mitgliedsfunktion, jedoch
  ist dies lediglich eine sperrige Übersetzung des englischen Begriffes
  {\em member functions}.
\stopitemize

Hier noch ein weiterer Tipp im Zusammenhang mit der Wahl der Identifier:
Bezeichner nur aus Großbuchstaben werden nicht empfohlen, da diese meist
für Makros im Kontext des Präprozessors verwendet werden.

\section[declaration]{Deklaration und Definition}

Bei einer Zuordnung von Name zu Typ handelt es sich um eine
{\em Deklaration}. Ein Name muss im Quelltext zuerst deklariert werden
bevor dieser verwendet werden kann.

Eine Deklaration wird in \cpp als Anweisung betrachtet. Das hat eine
Bedeutung, da damit auch der Geltungsbereich (siehe Abschnitt
\in[scope]) und speziell die Lebenszeit eines Speicherobjektes definiert
ist.

Enthält eine Deklaration alle Angaben, um den Namen zu benutzen, dann
spricht man von einer {\em Definition}. Handelt es sich dabei um die
Definition einer Variable, dann wird vom Compiler für diese Variable der
Speicher reserviert. Das bedeutet, dass es sich bei einer Definition um
eine spezielle Deklaration handelt. Hier einige Beispiele:

\startcpp
bool ready;  // Definition
char ch;  // Definition
extern int state;  // Deklaration
double result;  // Definition

double add(double, double);  // Deklaration
class User;  // Deklaration
\stopcpp

Für \type{ready}, \type{ch} und \type{result} kennt der Compiler nicht
nur den Namen und den Typ, sondern es wird vom Compiler auch
Speicherplatz reserviert werden, da alle Informationen vorhanden sind,
um diese Variable zu benutzen.

\type{extern} gibt an, dass es sich nur um eine Deklaration handelt und
die Definition an anderer Stelle (meist in einer anderen Datei) erfolgen
muss.

Bei \type{add} handelt es sich um die Deklaration einer Funktion, da nur
die Informationen bezüglich der Typen der Parameter sowie des
Rückgabewertes vorhanden sind, aber der Rumpf der Funktion fehlt. So
eine Funktionsdeklaration wird in \cpp als Prototyp bezeichnet. Damit
ist es möglich, dass der Compiler Funktionsaufrufe dieser Funktion
übersetzen kann, aber das Linken wird fehlschlagen, da der Linker nicht
weiß welche Adresse für den Funktionsaufruf eingesetzt werden muss.

Mittels \type{class User} wird dem Compiler mitgeteilt, dass es sich bei
\type{User} um einen Bezeichner handelt, der für einen
benutzerdefinierten Typ steht, aber es fehlt zu diesem Zeitpunkt noch
die Information wie dieser Typ konkret aussieht.

Da es sich bei Definitionen um spezielle Deklarationen handelt, werde
ich in weiterer Folge Deklaration als Überbegriff verwenden.

Es kann in einem Programm mehrere Deklarationen -- solange es sich nicht
um Definitionen handelt -- des gleichen Namens im gleichen
Geltungsbereich geben, dann müssen diese allerdings identisch sein. Der
folgende Quelltext ist korrekt:

\startcpp
extern int res;
extern int res;
\stopcpp

Allerdings wären die folgenden Deklarationen nicht korrekt, da die Typen
für den Namen \type{res} nicht übereinstimmen:

\startcpp
extern int res;
extern double res;
\stopcpp

Im Gegensatz dazu muss es genau eine Definition eines Namens im gesamten
Programm geben, außer der Bezeichner wird nicht verwendet oder es
handelt sich bei der Verwendung lediglich um einen Pointer, der nicht
dereferenziert wird (siehe später). Das folgende Beispiel

\startcpp
int res;
int res;
\stopcpp

oder auch

\startcpp
int res;
double res;
\stopcpp

sind beide syntaktisch falsch, da es eben mehrere Definitionen des
Bezeichners gibt.

Deklarationen werden meist mit einem Strichpunkt abgeschlossen. Die
Ausnahme bildet die Funktionsdefinition, die nach der geschlossenen
geschwungenen Klammer keinen Strichpunkt hat.

\section[structureofdeclarations]{Struktur einer Deklaration}

Den einfachsten Aufbau einer Deklaration haben wir uns schon angesehen,
nämlich Typangabe gefolgt von Bezeichner. Aber es gibt noch viel mehr.
Im letzten Abschnitt haben wir schon gesehen, dass es weitere
Schlüsselwörter wie \type{extern} gibt oder, dass es
Deklaratoroperatoren gibt und vieles mehr.

Vereinfacht sieht der Aufbau einer Deklaration folgendermaßen aus:

\startitemize[a][stopper=.]
\item
  Am Anfang kann ein Präfix (engl. {\em prefix}), wie z.B. \type{extern}
  oder \type{static} stehen.
\item
  Danach folgt verpflichtend der Basistyp wie z.B. \type{int},
  \type{User} oder \type{vector<string>}
\item
  Anschließend folgt ein Deklarator. Dieser besteht entweder aus einem
  Namen, einem Deklaratoroperator oder beidem.
\item
  Optional kann für Funktionen ein Suffix (engl. {\em suffix}) kommen
  wie z.B. \type{noexcept}.
\item
  Am Ende kann optional noch eine Initialisierung oder im Falle einer
  Funktionsdefinition der Funktionsrumpf stehen.
\stopitemize

Diese Struktur ist hier angeführt, sodass du einen allgemeinen Überblick
bekommst. Welche Möglichkeiten es detailliert gibt und wie du diese
einsetzen kannst, erfolgt in folgenden Kapiteln. Damit du dir trotzdem
etwas vorstellen kannst, wie so eine Deklaration aussehen kann, hier
noch ein kleines Beispiel: \mono{static int* p\{new int\{3\}\};} Eine
Erklärung dazu kommt später.

Die Verwendung von Deklaratoroperatoren ist komplex, deshalb beschränke
ich mich wieder auf die notwendigen und wichtigen Elemente. Eine
Erklärung der wichtigen Anwendungen dieser Deklaratoroperatoren erfolgt
ebenfalls später. Als Tipp empfehle ich, die Deklaration von rechts nach
links zu lesen:

\startitemize[packed]
\item
  Ein Pointer wird durch \type{*} gekennzeichnet. Ein einfaches Beispiel
  ist: \type{int* p}. Das bedeutet, dass \type{p} ein Pointer auf ein
  \type{int} ist.
\item
  Eine Referenz (konkret eine lvalue Referenz, siehe später) wird durch
  ein \type{&} gekennzeichnet. Ein Beispiel wäre:
  \mono{int& current\{value\}}. Das bedeutet, dass \type{current} ein
  anderer Name für die Variable \type{value} ist. Wird \type{current}
  verändert, wie z.B. mit \type{current = 3;} dann wurde eben der
  Speicherbereich von \type{value} verändert, d.h. auf den Wert 3
  gesetzt.
\item
  Ein Array wird mittels eckiger Klammern angegeben:
  \type{char first_name[20]} stellt ein Array dar, das sich aus 20
  Zeichen zusammensetzt.
\item
  Eine Funktion wird von \cpp durch ein folgendes Klammernpaar erkannt,
  wie z.B. \type{int sqrt();}.
\stopitemize

Wichtig: Außer bei Definitionen von Funktionen und Namensräumen ist eine
Deklaration immer mit einem Strichpunkt abzuschließen!

Prinzipiell kann man in \cpp mehrere Namen in einer Deklaration
deklarieren. Das ist nicht zu empfehlen und wird hier auch nicht
erklärt. Das bedeutet, dass wir jede Deklaration einer Variable in genau
einer Zeile schreiben, also so:

\startcpp
int age;
int count;
\stopcpp

\section[scope]{Geltungsbereich}

Nachdem wir jetzt wissen was eine Deklaration ist und außerdem den
Aufbau einer Deklaration kennen, stellt sich weiters die Frage wo so
eine Deklaration gültig ist. Der Geltungsbereich (engl. {\em scope},
eingedeutscht Scope) legt eben fest in welchem Bereich der eingeführte
Bezeichner gültig ist.

Als generelle Regel lässt sich sagen, dass ein Bezeichner vom Beginn der
Deklaration bis zum Ende des Blockes, in dem er deklariert wird, seine
Gültigkeit behält. Von dieser Regel gibt es drei Ausnahmen:

\startitemize
\item
  Bezeichner, die Instanzvariablen oder Methoden einer Klasse
  darstellen, sind in der gesamten Klassendeklaration gültig. Wir werden
  uns das später noch ansehen.
\item
  Globale Bezeichner sind in keinem Block enthalten und behalten ihre
  Gültigkeit bis zum Ende der Datei. Auch das werden wir uns später noch
  ansehen.
\item
  Bezeichner, die innerhalb der runden Klammern einer \type{for},
  \type{while} oder \type{switch} Anweisung oder einer
  Funktionsdefinition deklariert wurden erstrecken sich vom Beginn der
  Deklaration des Bezeichners bis zum Ende des folgenden Blockes. Ein
  Beispiel für so eine Deklaration haben wir schon im Abschnitt
  \in[exasort] auf der Seite \at[exasort] gesehen.
\stopitemize

Bezeichner können durch andere Deklarationen überschattet werden (engl.
{\em shadowing}). Das kann durch verschachtelte Geltungsbereiche
erreicht werden. Teste dies in folgendem Programm:

\startcpp
// scope.cpp
#include <iostream>

using namespace std;

int x;

int main() {
    cout << "global x: " << x << endl;
    
    int x;

    cout << "local x: " << x << endl;
}
\stopcpp

Die lokale Variable \type{x} überschattet die globale Variable \type{x}.
Bei mir kommt es deshalb zu folgender Ausgabe:

\startsh
global x: 0
local x: 134514675
\stopsh

Was hier zu sehen ist:

\startitemize[packed]
\item
  Zuerst wird eine globale Variable \type{x} definiert, aber nicht
  explizit initialisiert.
\item
  In \type{main} erfolgt die Ausgabe der globalen Variable \type{x}. Es
  handelt sich hier noch um die globale Variable, da die Gültigkeit der
  lokalen Variable erst mit deren Deklaration beginnt. Weiters sieht
  man, dass der Wert 0 ausgegeben wird. Das hat damit zu tun, dass
  globale Variablen mit dem Nullwert initialisiert werden.
\item
  Danach folgt die Deklaration einer lokalen Variable \type{x}. Auch
  diese Variable wird nicht initialisiert. Dies sieht man gut bei der
  nachfolgenden Ausgabe dieser lokalen Variable: es wird der Wert der
  Speicherstelle ausgegeben, bei mir eben 134514675.
\stopitemize

Hänge an das Programm die folgenden Anweisungen an:

\startcpp
x = 1;

{
    int x;
    cout << "x in block: " << x << endl;
}

cout << "local x (after block): " << x << endl;
\stopcpp

Jetzt kommt es bei mir zu folgender Ausgabe:

\startsh
global x: 0
local x: 134514787
x in block: -1076411604
local x (after block): 1
\stopsh

Analysieren wir diese Ausgabe:

\startitemize[packed]
\item
  Wir bemerken, dass die Ausgabe einer uninitialisierten Variable nicht
  immer den gleichen Wert aufweisen muss!
\item
  Als Nächstes bemerken wir, dass eine lokale Variable in einem Block
  eine außenliegende Variable mit dem gleichen Namen
  \quotation{überschattet}. Damit haben die Änderungen der lokalen
  Variable im Block keine Auswirkungen auf die Variable außerhalb des
  Blockes.
\stopitemize

Erweitern wir das Programm nochmals durch Anhängen der folgenden
Anweisungen:

\startcpp
for (int x{2}; x < 3; ++x) {
    cout << "x in statement: " << x << endl;
}

cout << "local x (after assignment): " << x << endl;
\stopcpp

Teste das Programm! Du siehst, dass die Deklaration der Variable
innerhalb der runden Klammern der \type{for} Anweisung für die gesamte
\type{for} Anweisung gilt, aber nicht außerhalb. Sehr praktisch.

Jetzt wollen wir das Beispiel über die Geltungsbereiche abschließen,
indem wir eine neue Funktion definieren und diese aufrufen. Füge deshalb
die nachfolgende Funktionsdefinition vor \type{main} ein:

\startcpp
void func(int y) {
    cout << "x in func (global): " << x << endl;
    ++y;
    cout << "y in func (local): " << y << endl;
}
\stopcpp

Hänge weiters die folgenden Anweisungen in \type{main} hinten an:

\startcpp
int y{1};

cout << "local y (before function call): " << y << endl;

func(y);

cout << "local y (after function call): " << y << endl;
\stopcpp

Der relevante Teil der Ausgabe sieht folgendermaßen aus:

\startsh
local y (before function call): 1
x in func (global): 0
y in func (local): 2
local y (after function call): 1
\stopsh

Hier sehen wir zwei Aspekte:

\startitemize[packed]
\item
  Bei \type{y} innerhalb der Funktion handelt es sich um einen anderen
  Bezeichner als \type{y} innerhalb von \type{main}. Da eine Variable
  standardmäßig {\em kopiert} wird, hat eine Änderung innerhalb der
  Funktion \type{func} keine Auswirkung auf die Variable \type{y}
  innerhalb von \type{main}.
\item
  In \type{func} wird mit dem Namen \type{x} auf die globale Variable
  \type{x} zugegriffen und nicht auf die lokale Variable \type{x} in der
  Funktion \type{main}.
\stopitemize

\section[initialization]{Initialisierung}

\subsection[arten-der-initialisierung]{Arten der Initialisierung}

An sich gibt es vier Möglichkeiten, wie man in \cpp Variablen
initialisieren kann. Hier folgen Beispiele für diese verschiedenen
Varianten:

\startcpp
int i1{1};
int i2 = {1};
int i3 = 1;
int i4(1);
\stopcpp

Die erste Variante wird als \quotation{uniform initialization} also die
einheitliche Initialisierung bezeichnet, da sie weitgehend überall
verwendet werden kann, wie zum Beispiel bei Arrays, Strukturen, Klassen
und Templates.

Die erste unterscheidet sich von der zweiten Variante geringfügig, wenn
ein benutzerdefinierter Datentyp über einen Konstruktor verfügt, der mit
\type{explicit} markiert ist. Das werden wir uns später noch ansehen.

Die erste und zweite Variante haben die wichtige Eigenschaft, dass sie
keine Konvertierungen zulassen, die nicht werterhaltend sind.
Werterhaltend bedeutet, dass der Wert gleich dem ursprünglichen Wert
sein muss, wenn er wieder in den ursprünglichen Typ gewandelt wird.

Das folgende Beispiel demonstriert dies, da die dritte und vierte
Variante nicht-werterhaltende Konvertierungen durchführen:

\startcpp
int i5 = 2.5;  // 2.5 hat den Typ double
int i6(0.5);
\stopcpp

Damit wird \type{i5} mit dem Wert \type{2} und \type{i6} mit dem Wert
\type{0} initialisiert. Das wird als {\em narrowing} (dt. einengen)
bezeichnet, da diese Konvertierung nicht werterhaltend ist. Genau das
ist aber hier der Fall, denn würde man den Wert von \type{i5} wieder in
einen \type{double} wandeln, dann erhält man lediglich den Wert
\type{2.0}.

Es handelt sich hierbei wahrscheinlich um einen Programmierfehler. Warum
sollte man eine \type{int} Variable mit \type{2.5} initialisieren
wollen? Man würde in diesem konkreten Fall doch einfach \type{2}
schreiben, nicht wahr? Hätte man stattdessen die einheitliche
Initialisierung verwendet, hätte der Compiler eine entsprechende
Fehlermeldung erzeugt.

Die vierte Form hat einen weiteren Nachteil. Nehmen wir einmal an, dass
wir die folgende Deklaration \type{int i6();} haben, mit der Absicht
eine Variable \type{i6} vom Typ \type{int} anzulegen und diese mit dem
Nullwert zu initialisieren. In Wirklichkeit wird der Compiler dies als
eine Funktionsdeklaration interpretieren, die eine Funktion mit dem
Namen \type{i6} einführt, die keine Parameter besitzt und einen
\type{int} zurückliefert!

Das bedeutet, dass die dritte und die vierte Variante jeweils zu
vermeiden sind, wenn sich das erreichen lässt, da es hier zu sogenannten
einengenden impliziten Konvertierungen kommen kann und diese beiden
Varianten von der Syntax von \cpp auch nicht {\em überall} verwendet
werden können.

\subsection[ausnahmen-zur-einheitlichen-initialisierung]{Ausnahmen zur
einheitlichen Initialisierung}

Von diesen vier Möglichkeiten kann nur die erste Variante weitgehend
überall eingesetzt werden. Es gibt allerdings zwei Ausnahmen:

\startitemize
\item
  Wenn eine \type{auto} Deklaration verwendet wird. Eine \type{auto}
  Deklaration kann sehr praktisch sein, da der Compiler den Typ
  selbständig ermittelt. Das reduziert die Redundanz und andererseits
  ist es manchmal gar nicht so einfach, den richtigen Typ bei der
  Verwendung der Standardbibliothek herauszufinden. Schreibe das
  folgende Beispiel und probiere es aus:

  \startcpp
  // auto.cpp
  #include <iostream>
  #include <vector>
  #include <algorithm>

  using namespace std;

  int main() {
      auto words = vector<string>{"prolog", "java", "lisp",
                                  "python", "c++"};

      sort(words.begin(), words.end());

      for (int i{0}; i < words.size(); ++i) {
          cout << words[i] << endl;
      }
  }
  \stopcpp

  Hier verwenden wir eine lokale Variable \type{words}, die beim Anlegen
  mit einer Liste initialisiert wird. Diese in geschwungenen Klammern
  eingeschlossene Liste von Werten entspricht dem Typ
  \type{std::initializer_list} und wird in der Standardbibliothek oft
  verwendet und kann auch für eigene Zwecke ebenfalls verwendet werden.

  Da es sich bei solch einer Initialisierungsliste um diesen Typ
  \type{std::initializer_list<int>} handelt, erhält man mit
  \mono{auto i\{1\};} auch nicht das meist erwünschte Ergebnis,
  äquivalent zu \mono{int i\{1\}}, sondern
  \mono{std::initializer_list<int> i\{1\}}! Das bedeutet, dass bei
  Verwendung von \type{auto} in der Regel die Initialisierungsvariante
  mit \type{=} verwendet werden muss!

  Der Rest des Beispiels ist wieder gleich. Allerdings wollen wir die
  Gelegenheit gleich nutzen, um die Schleife mit der Ausgabe ebenfalls
  umzugestalten. Ersetze dazu die Schleife durch das folgende Konstrukt
  und teste:

  \startcpp
  for (string elem : words) {
      cout << elem << endl;
  }
  \stopcpp

  Es handelt sich hierbei um eine \quotation{for each} Schleife, wie
  diese auch von anderen Programmiersprachen bekannt sein dürfte: Es
  wird über alle Elemente des Vektors \type{words} iteriert und je
  Schleifendurchgang erhält die Laufvariable das jeweils aktuelle
  Element.

  So, jetzt können wir auch dieses Beispiel noch verbessern. Dem
  Compiler ist der Typ von \type{words} bekannt, nämlich
  \type{vector<string>}. Daher weiß der Compiler auch, dass der Typ von
  \type{elem} nur \type{string} sein kann und das wird von ihm auch
  überprüft. Das kannst du leicht überprüfen indem du einen anderen Typ
  für \type{elem} angibst, wie z.B. \type{int}. Du wirst einen
  Syntaxfehler erhalten.

  Da dem Compiler bekannt ist, um welchen Typ es sich handelt, kann er
  diesen auch selbst ermitteln, womit wir wiederum das Schlüsselwort
  \type{auto} zum Einsatz bringen können:

  \startcpp
  for (auto elem : words) {
      cout << elem << endl;
  }
  \stopcpp

  Nicht schlecht, oder? Verwenden wir in solch einem Fall \type{auto}
  hat das den weiteren Vorteil, dass wir den Typ von \type{words} z.B.
  auf \type{vector<int>} ändern könnten. Die \type{for} Schleife müsste
  nicht verändert werden!

  Eine andere Variante wäre eine Referenz für die Schleifenvariable zu
  verwenden. Füge die folgende Schleife vor die schon bestehende
  Schleife ein und teste:

  \startcpp
  for (auto& elem : words) {
      elem = elem + "!";
  }
  \stopcpp

  Damit wird die Schleifenvariable als neuer Name für das jeweils
  aktuelle Schleifenobjekt verwendet, womit das aktuelle Schleifenobjekt
  direkt in unserem \type{vector} verändert werden kann. Das bedeutet,
  dass das Schleifenobjekt nicht kopiert wird.

  Es gibt noch einen Grund nicht kopieren zu wollen, wenn nämlich das
  Kopieren einen beträchtlichen Aufwand darstellt, da das zu kopierende
  Schleifenobjekt groß ist. Unter Umständen will man allerdings das
  Schleifenobjekt nicht verändern und sicherstellen, dass es zu keiner
  Veränderung kommt, dann schreibt man das Schlüsselwort \type{const}
  direkt vor \type{auto&}. Das könnte dann folgendermaßen aussehen:

  \startcpp
  for (const auto& elem : words) {
      cout << elem << endl;
  }
  \stopcpp
\item
  Kommen wir jetzt zum zweiten Fall wo \type{auto} nicht einzusetzen
  ist. Bei dem Typ der Variable handelt es sich um einen
  benutzerdefinierten Typ, der sowohl einen Konstruktor mit genau den
  angegebenen Initialisierungswerten hat und außerdem einen Konstruktor
  besitzt, der eine Initialisierungsliste als Parameter erwartet.

  Wie wir schon gesehen haben, kann man einen Vektor mithilfe einer
  Initialisierungsliste mit Werten initialisieren, wie z.B.
  \mono{vector<int> nums\{1, 2, 3, 4\}}. Andererseits gibt es auch einen
  Konstruktor mit dem man den Vektor mit einer bestimmten Anzahl von
  Nullwerten initialisieren kann. Angenommen man will einen Vektor mit
  10 Elementen anlegen, dann kann natürlich auch
  \mono{vector<int> nums\{10\}} nicht funktionieren, da man damit einen
  Vektor mit einem Element anlegt, das den Wert 10 hat.

  In diesem Fall ist es notwendig auf die explizite Konstruktorform
  zurückzugreifen: \type{vector<int> nums(10)} legt einen Vektor mit 10
  Elementen an, die jeweils mit 0 belegt sind.

  Das bedeutet, dass die Klasse \type{vector} sowohl einen Konstruktor
  hat, der sich eine Initialisierungsliste erwartet als auch einen
  Konstruktor, der sich eine ganze Zahl erwartet. Das Verhalten der
  beiden Konstruktoren ist eben unterschiedlich und genau wie im
  vorhergehenden Absatz beschrieben.

  Beachte, dass gerade die Notation mit den runden Klammern seine Tücken
  hat, die wir uns im Abschnitt \in[functiondeclaration] über Funktionen
  noch ansehen werden.
\stopitemize

\subsection[fehlende-initialisierungen]{Fehlende Initialisierungen}

Fehlt bei einer Definition einer Variable eines eingebauten Datentyps
die Initialisierung, dann hängt es davon ab, um welche Art der
Initialisierung es sich handelt. Bei benutzerdefinierten Datentypen wird
immer der Konstruktor aufgerufen.

Es gelten die beiden nachfolgenden Regeln:

\startitemize
\item
  Eine globale Variable, eine Variable aus einem Namespace, eine lokale
  \type{static} Variable oder eine \type{static} Member-Variable wird
  jeweils mit dem entsprechenden Nullwert initialisiert. Bei einem
  benutzerdefiniertem Datentyp entspricht dies dem Aufruf des
  Defaultkonstruktors (siehe Abschnitt \in[constructor]).

  Solch eine Initialisierung wird durchgeführt bevor die Funktion
  \type{main} gestartet wird und findet in der Reihenfolge der
  Deklaration statt. Zwischen verschiedenen Übersetzungseinheiten (siehe
  Abschnitt \in[translationunit]) ist die Reihenfolge der
  Initialisierung nicht definiert.
\item
  Eine lokale Variable oder ein Speicherobjekt, das am Heap angelegt
  wurde, wird hingegen nicht initialisiert.
\stopitemize

\section[speicherobjekte-und-werte]{Speicherobjekte und Werte}

\subsection[storageobject]{Speicherobjekt}

Wir wollen hier die Grundlagen schaffen, um die spätere Behandlung von
Referenzen besser verstehen zu können.

Dazu betrachten wir vorerst den Begriff \quotation{object} (dt. Objekt,
Speicherobjekt). Es handelt sich hierbei {\em nicht} um ein Objekt im
Sinne der Objektorientierung! Vielmehr bezeichnet dieser Begriff in
\cpp einen zusammenhängenden Bereich im Speicher.

Ein Speicherobjekt liegt im Speicher, beginnt an einer bestimmten
Adresse und hat eine bestimmte Größe. Die Adresse ist die eindeutige
Möglichkeit auf dieses Speicherobjekt zuzugreifen.

Über das Typkonzept von \cpp wird diesem Speicherobjekt eine bestimmte
Bedeutung zugewiesen und damit sind wiederum die Operationen festgelegt
wie mit diesem Speicherobjekt verfahren werden kann.

Haben wir zum Beispiel eine Variable vom Typ \type{int}, die mittels
\type{int result;} definiert worden ist, dann kann der Compiler diese
Variable in einem Speicherobjekt anlegen. Da diesem Speicherobjekt jetzt
über das Typsystem der Typ \type{int} zugewiesen ist, sind hiermit auch
die Operationen eindeutig definiert, wie auf die Speicherstelle
zugegriffen werden kann. Außerdem ist natürlich auch klar, dass man
daher mit den üblichen arithmetischen Operationen rechnen kann.

Jedes Objekt hat eine Lebenszeit. Das bedeutet, dass es erzeugt wird,
dann wird es verwendet und am Ende wird der Speicher wieder freigegeben.
Die Lebenszeit eines Objektes beginnt, wenn der Konstruktor
abgeschlossen ist und endet, wenn der Destruktor mit der Ausführung
beginnt. Ein Konstruktor ist eine Funktion, die für die Initialisierung
eines Objektes zuständig ist und ein Destruktor ist eine Funktion, die
die Beendigungsaktionen durchführt. Hat der Typ des Objektes, wie zum
Beispiel ein \type{int}, keinen Konstruktor oder Destruktor, dann wird
dies so betrachtet als hätte dieser Typ einen leeren Konstruktor oder
einen leeren Destruktor.

In \cpp gibt es verschiedene Arten von Lebenszeit. Die folgenden
Erläuterungen greifen bezüglich der Begriffe vor und müssen beim ersten
Durcharbeiten nicht verstanden werden:

\startdescription{Automatic}
  Ein Objekt mit der Lebenszeit {\em automatic}, wird bei der
  Deklaration erzeugt und beendet sich, wenn der Geltungsbereich der
  Deklaration endet. Lokale Variable haben diese Lebenszeit.

  \startcpp
  void f() {
      int i{1};  // Lebenszeit beginnt
      {
          int j{1};  // Lebenszeit beginnt
      }  // hier endet die Lebenszeit von j
  }  // hier endet die Lebenszeit von i
  \stopcpp
\stopdescription

\startdescription{Static}
  {\em Static} bedeutet, dass das Objekt genau einmal initialisiert wird
  und seine Lebenszeit erst beim Beenden des Programmes verliert. Dazu
  zählen globale Variable, Variable in Namensräumen, \type{static}
  deklarierte Variable in Funktionen und \type{static} Member-Variablen.

  \startcpp
  // Lebenszeit beginnt mit dem Start des Programmes
  int i;  // globale Variable; initialisiert mit 0;

  int main() {

  }  // hier endet die Lebenszeit von i
  \stopcpp
\stopdescription

\startdescription{Free store}
  Als {\em free store} wird in \cpp der Heap bezeichnet. Objekte, die
  dort liegen müssen explizit angefordert und explizit wieder
  freigegeben werden. Dazu gibt es in \cpp die Operatoren \type{new} und
  \type{delete}.

  \startcpp
  int* p{new int{1}};  // Lebenszeit beginnt

  delete p;  // Lebenszeit endet
  \stopcpp
\stopdescription

\startdescription{Temporary objects}
  Temporäre Objekte ({\em temporary objects}) entstehen als
  Zwischenschritt in der Auswertung eines Ausdruckes. Ihre Lebenszeit
  endet entweder mit dem Ende des vollständigen Ausdruckes in dem diese
  erzeugt worden sind oder mit der Lebenszeit der Referenz, wenn dieses
  temporäre Objekt an eine Referenz gebunden worden ist.

  \startcpp
  int i;
  int j;
  i = (i + j) * 4;
  \stopcpp

  Temporäre Objekte sind in diesem Beispiel der Wert von \type{i + j}
  und der Wert von \type{(i + j) * 4}. Der vollständige Ausdruck ist
  \type{i = (i + j) * 4}. Mit dem Ende dieses vollständigen Ausdruckes
  beenden sich auch die Lebenszeiten aller enthaltenen temporären
  Objekte. Natürlich macht es in diesem Fall keinen Sinn sich darüber
  Gedanken zu machen, da es unerheblich ist, wann die Lebenszeit von
  derartigen \type{int} Speicherobjekten endet. Handelt es sich aber um
  benutzerdefinierte Datentypen, die einen Destruktor haben, dann ist es
  unter Umständen sehr Wohl von Interesse zu wissen, wann der Destruktor
  aufgerufen wird, da dieser Destruktor Nebeneffekte haben kann.
\stopdescription

\startdescription{Thread-local objects}
  Der Vollständigkeit halber erwähne ich auch noch die Objekte, deren
  Lebenszeit an die Lebenszeit des Threads gebunden ist.
\stopdescription

\subsection[values]{Werte}

Jeder Ausdruck hat einen Wert (engl. {\em value}). Man unterscheidet
zwei Arten von Werten: {\em lvalue} und {\em rvalue}.

Als {\em lvalue} wird ein Ausdruck verstanden, der auf ein Objekt
verweist, das über den Ausdruck hinaus erhalten bleibt. Ein lvalue hat
seinen Namen davon, dass dieser auf der linken Seite einer Zuweisung
(\quotation{left-hand side}, abgekürzt lhs) vorkommen kann. Es gibt
allerdings auch lvalues, die nicht auf der linken Seite vorkommen
können, da diese \type{const} sind und hiermit nicht veränderbar.

Als Faustregel gilt: Ein Ausdruck, von dem du mittels des \type{&}
Operators eine Adresse ermitteln kannst, ist ein lvalue.

Das folgende Beispiel enthält auch Pointer (Zeiger), die erst später
beschrieben werden:

\startcpp
int i{};
int* p{&i};  // Pointer p wird mit Adresse von i initialisiert
const int j{1};

i = 2;
*p = 3;  // Objekt auf das p verweist wird mit 3 belegt
j = 4; // Compilerfehler, da j konstant ist
\stopcpp

Hier ist \type{i} ein lvalue, da \type{i} ein Name (auch ein Ausdruck)
ist, der direkt auf ein Objekt verweist. \type{*p} ist ebenfalls ein
lvalue, da in das Objekt der Wert 3 geschrieben wird, auf das \type{p}
verweist. \type{j} ist zwar ein lvalue, kann aber nicht verändert
werden, da dieser konstant ist.

Beachte, dass du die Definition der Konstante \type{j} auch
folgendermaßen anschreiben kannst:

\startcpp
int const j{1};
\stopcpp

Damit funktioniert die \quotation{lese von rechts nach links}-Regel in
100\letterpercent{} der Fälle. Allerdings ist die andere Schreibweise
gebräuchlicher.

Das Gegenstück zu einem lvalue ist ein {\em rvalue}. Ein rvalue ist ein
Ausdruck, der auf ein Objekt verweist, das nicht über den Ausdruck
hinaus erhalten bleibt. Vereinfacht kann man sagen, dass ein rvalue ein
Ausdruck ist, der kein lvalue ist.

Betrachten wir dazu den folgenden Quelltext:

\startcpp
int i;
int j;

i = (i + j) * 4;
\stopcpp

Hier handelt es sich bei dem Ergebnis von \type{i + j} um einen rvalue
und auch das Ergebnis von \type{(i + j) * 4} ist ein rvalue. \type{i}
auf der linken Seite von \type{=} ist jedoch ein lvalue.

Kann man von einem Ausdruck eine Adresse bestimmen, dann handelt es sich
typischerweise um einen lvalue, ansonsten um einen rvalue. Das bedeutet
aber auch, dass für einen gegebenen Typ \type{T} es sowohl lvalues von
dem Typ \type{T} als auch rvalues von dem Typ \type{T} geben kann!

\section[constants]{Konstanten}

Konstanten sind Bezeichner, deren Werte nicht mehr verändert werden
können. Dadurch ist es notwendig, dass Konstanten bei der Definition
auch initialisiert werden müssen.

Es gibt in \cpp zwei Arten wie man Konstanten definieren kann:

\startitemize
\item
  Durch die Verwendung des Schlüsselwortes \type{const}, wie wir dies
  schon kennengelernt haben. Hier wollen wir den Schwerpunkt allerdings
  auf die Initialisierung legen.

  Die Initialisierung wird bei der Verwendung von \type{const} zur
  Laufzeit vorgenommen und kann in weiterer Folge nicht mehr verändert
  werden. Das folgende Beispiel demonstriert dies:

  \startcpp
  // constants.cpp
  #include <iostream>

  using namespace std;

  int main() {
      int min;
      int size;

      cout << "Minimum: ";
      cin >> min;
      cout << "Größe: ";
      cin >> size;

      const int max{min + size};
      cout << "Maximum = " << max << endl;
      //max = 3;  // Fehler!
  }
  \stopcpp
\item
  Die zweite Möglichkeit besteht in der Verwendung des Schlüsselwortes
  \type{constexpr}, das so viel wie \quotation{constant expression} --
  also konstanter Ausdruck -- bedeutet.

  Der Unterschied zu \type{const} liegt darin, dass die Initialisierung
  nicht zur Laufzeit, sondern zur Übersetzungszeit vorgenommen wird. Das
  bedeutet, dass der Compiler den Wert der Konstante bestimmt. Dazu muss
  der gesamte Ausdruck der Initialisierung entweder aus Literalen,
  \type{constexpr}-Konstanten oder \type{constexpr}-Funktionen
  zusammengesetzt sein. Als Ausnahme dürfen auch \type{const}-Konstanten
  verwendet werden, soferne diese mit einem konstanten Ausdruck
  initialisiert worden sind.

  Hänge folgende Codezeilen an dein Programm:

  \startcpp
  const int months{12};
  constexpr int days_per_month{30};
  constexpr int days_per_year{months * days_per_month};
  cout << "Ein Bankenjahr hat " << days_per_year << " Tage"
       << endl;
  \stopcpp

  Es gibt einen weiteren Unterschied zu der Verwendung von \type{const},
  da es sich dabei prinzipiell nicht um Speicherobjekte handelt. Das
  bedeutet, dass in der obigen Ausgabe anstatt \type{days_per_year} vom
  Compiler direkt der berechnete Wert 360 eingesetzt wird.

  Erst wenn wir den Adressoperator explizit verwenden, wird ein
  Speicherobjekt angelegt. Hänge die folgenden Codezeilen an und es wird
  ein Speicherobjekt angelegt. Die Ausgabe erfolgt natürlich wieder wie
  erwartet:

  \startcpp
  const int* p{&days_per_year};
  cout <<  "Ein Bankenjahr hat " << *p << " Tage" << endl;
  \stopcpp
\stopitemize

\section[implicitconv]{Implizite Konvertierungen}

Dieser Abschnitt erklärt die Konvertierungen, die von \cpp implizit
vorgenommen werden. Da die fundamentalen Datentypen noch nicht im Detail
erläutert worden sind, kann dieser Abschnitt beim ersten Lesen
\quotation{überflogen} werden. Spätestens beim Durcharbeiten der
einzelnen fundamentalen Datentypen kann dieser Abschnitt zwecks
genaueren Verständnis nochmals aufgesucht werden.

Von \cpp werden implizite Konvertierungen selbständig vorgenommen.
Nehmen wir an wir haben eine Zuweisung der Form \type{x = expr}, wobei
es sich bei \type{expr} um einen arithmetischen Ausdruck handelt. Die
nachfolgenden Ausführungen gelten analog auch nur für den Ausdruck
\type{expr} alleine. \cpp geht folgendermaßen vor:

\startitemize[a][stopper=.]
\item
  Zuerst wird eine Aufweitung der integralen Datentypen (engl.
  {\em integral promotion}, eingedeutscht Promotion) vorgenommen. Das
  bedeutet, dass \type{int}s aus kleineren Datentypen erzeugt werden und
  hat den Sinn, dass die Operanden in das \quotation{natürliche} Format
  für arithmetische Operationen gebracht werden.

  Im speziellen werden die folgenden Promotions durchgeführt:

  \startitemize
  \item
    Ein \type{char}, \type{signed char}, \type{unsigned char},
    \type{short int} oder \type{unsigned short int} wird in einen
    \type{int} konvertiert, wenn der \type{int} alle Werte
    repräsentieren kann, anderenfalls in einen \type{unsigned int}.
  \item
    Ein \type{char16_t}, \type{char32_t}, \type{wchar_t} oder ein
    einfacher Enumerationstyp (kein \type{enum class}) werden zum ersten
    der folgenden Typen konvertiert, der alle Werte repräsentieren kann:
    \type{int}, \type{unsigned int}, \type{long}, \type{unsigned long},
    \type{unsigned long long}.
  \item
    Ein \type{bool} wird zu einem \type{int} konvertiert.
  \stopitemize

  Schauen wir uns dazu wieder ein Beispiel an:

  \startcpp
  // promotions.cpp
  #include <iostream>

  using namespace std;

  int main() {
      char a{'0'};
      char b{'0'};

      cout << "a = " << a << "; sizeof(a) = " << sizeof(a)
           << endl;
      cout << "b = " << b << "; sizeof(b) = " << sizeof(b)
           << endl;
       cout << "a + b = " << a + b << "; sizeof(a + b) = "
            << sizeof(a + b) << endl;
  }
  \stopcpp

  Als Ergebnis erhältst du dann:

  \startsh
  a = 0; sizeof(a) = 1
  b = 0; sizeof(b) = 1
  a + b = 96; sizeof(a + b) = 4
  \stopsh

  Obwohl die Größe eines \type{char} per Definition in \cpp immer 1 ist,
  ist die Größe von \type{a + b} gleich 4. Das liegt eben daran, dass
  ein \type{char} mittels Promotion zu einem \type{int} aufgeweitet
  wird. Im Bereich der ganzen Zahlen wird danach die Addition
  durchgeführt, womit das Ergebnis dann auch die Größe eines \type{int}s
  aufweist.

  Als Ergebnis kommt 96 heraus, weil das dezimale Äquivalent des
  Zeichens \type{'0'} in der ASCII Kodierung der Wert 48 ist.
\item
  Danach werden Konvertierungen vorgenommen, um die Typen eines
  Ausdruckes auf einen gemeinsamen Typ zu bringen. Die exakten Regeln
  sind am Besten in einer Referenz nachzulesen, aber das zugrunde
  liegende Prinzip ist, dass jeweils auf den nächst größeren Datentyp
  konvertiert wird.

  Das folgende Beispiel in der Datei \type{conversions.cpp} soll dies
  demonstrieren:

  \startcpp
  #include <iostream>

  using namespace std;

  int main() {
      long long int ll{};
      char c{};

      cout << "sizeof(ll) = " << sizeof(ll) << endl;
      cout << "sizeof(c) = " << sizeof(c) << endl;
      cout << "sizeof(ll + c) = " << sizeof(ll + c) << endl;
  }
  \stopcpp

  Auf meinem System kommt es zu folgender Ausgabe:

  \startsh
  sizeof(ll) = 8
  sizeof(c) = 1
  sizeof(ll + c) = 8
  \stopsh

  Du siehst, dass auf meinem System ein \type{long long int} die Größe 8
  hat, während die Größe eines \type{char} eben 1 ist. Das Ergebnis der
  Addition hat ebenfalls die Größe 8, weil eben der kleinere Typ
  \type{char} zuerst auf einen \type{int} aufgeweitet wird und danach
  per impliziter Konversion in einen \type{long long int} konvertiert
  worden ist.
\item
  Als Spezialfall von b. wird der Typ des Werts des Ausdruckes in den
  Typ von der Variable der linken Seite der Zuweisung gebracht. Hier
  muss man speziell aufpassen! Hänge folgende Codezeilen an das Programm
  an:

  \startcpp
  int i{};
  i = 3.5;
  cout << i << endl;
  \stopcpp

  Das Programm übersetzt einwandfrei und liefert selbstverständlich den
  Wert \type{3} auf der Ausgabe. Gegen eine einengende Konvertierung bei
  der Initialisierung können wir uns mit der einheitlichen
  Initialisierung wehren, aber bei einer Zuweisung kommt es -- aufgrund
  der Wurzeln in der Programmiersprache -- zu impliziten
  Konvertierungen, die in der Regel nicht gewollt sind.

  Leider kann es noch viel unangenehmer werden. Füge folgende Zeile an
  das Programm an:

  \startcpp
  char c = 128;
  cout << c << endl;
  \stopcpp

  Auf einem System mit 8 Bit vorzeichenbehafteten \type{char}s, kommt es
  zu einem Überlauf und zu undefiniertem Verhalten. Daher sollten
  derartige Konvertierungen weitgehend vermieden werden. Kann dies nicht
  erreicht werden, dann sollte eine explizite Konvertierung zwecks
  Dokumentation vorgenommen werden.

  Helfen kann man sich, indem man den Compiler beauftragt, eine Warnung
  anzuzeigen, wenn eine derartige implizite, nicht werterhaltende
  Konvertierung vorgenommen wird. Für die entsprechende Option für den
  Compiler siehe Anhang \in[compilation].
\stopitemize

\section[automatische-typbestimmung]{Automatische Typbestimmung}

Es gibt in \cpp zwei Möglichkeiten wie man den Typ im Zuge einer
Deklaration von \cpp bestimmen lassen kann:

\startitemize
\item
  Einerseits kann das schon bekannte \type{auto} verwendet werden (siehe
  Abschnitt \in[initialization]). Der große Vorteil tritt zutage, wenn
  \type{auto} im Zusammenhang mit Templates verwendet wird.

  In \cpp gibt es -- wie in vielen anderen Programmiersprachen auch --
  ein Iteratorkonzept. Ein Vektor kann nicht nur mittels einer
  Zählschleife und Zugriff über den Index oder einer \quotation{for
  each} Schleife, sondern auch mittels eines Iterators durchlaufen
  werden.

  Teste das folgende Programm:

  \startcpp
  // iterators.cpp
  #include <iostream>
  #include <vector>

  using namespace std;

  int main() {
      vector<int> v{1, 2, 3, 4, 5};

      for (vector<int>::iterator it{v.begin()};
           it != v.end(); ++it)
          cout << *it << endl;
  }
  \stopcpp

  Was passiert? Zuerst wird ein Vektor mit \type{int} Werten angelegt
  und initialisiert. Dann werden mittels einer \type{for} Schleife alle
  Werte des Vektors durchlaufen und ausgegeben.

  Der interessante Aspekt liegt in den Iteratoren. Am Beginn der
  Schleife wird ein Iterator mittels \type{begin()} initialisiert.
  Dieser Iterator zeigt an den Beginn des Vektors. Solange der Iterator
  nicht an das Ende zeigt, wird der Schleifenrumpf betreten und am Ende
  der Iterator weitergesetzt.

  Das Komplizierte an dieser Schleife ist, den korrekten Typ des
  Iterators zu bestimmen und anzuschreiben. Dazu muss man entweder das
  Wissen haben, wie dieser Typ aussieht oder die Dokumentation zurate
  ziehen. Unabhängig davon ist es jedoch mühselig diesen anzuschreiben.
  Hier kommt das Schlüsselwort \type{auto} gerade recht. Ändere den
  Schleifenkopf entsprechend ab:

  \startcpp
  for (auto it = v.begin(); it != v.end(); ++it)
  \stopcpp

  Das sieht ja schon viel einfacher aus! Der Compiler kennt ja den
  Rückgabewert von \type{vector::begin()} und deshalb können wir den
  Compiler den Typ selbständig bestimmen lassen. Beachte lediglich, dass
  die Initialisierung mittels der geschwungenen Klammern -- wie schon
  besprochen -- durch ein Gleichheitszeichen ersetzt wurde.
\item
  Die zweite Möglichkeit besteht darin, \type{decltype()} zu verwenden,
  das hauptsächlich bei der Deklaration von Templates Verwendung findet.
  Für eine Verwendung verweise ich auf Abschnitt
  \in[functiontemplates] auf der Seite \at[functiontemplates].
\stopitemize

\section[using]{\type{using} Direktive und Deklaration}

Das Schlüsselwort \type{using} wird in drei verschiedenen Arten
verwendet:

\startitemize
\item
  Die erste Variante kennen wir schon (\type{using}-Direktive), die
  bewirkt, dass {\em alle} Bezeichner des angegebenen Namensraumes im
  aktuellen Geltungsbereich zugreifbar sind. Obwohl alle Bezeichner
  zugreifbar sind, werden diese nicht zum lokalen Scope hinzugefügt.

  Das klassische Beispiel ist \type{using namespace std;}, das jedoch
  {\bf nie} in Headerdateien (siehe Abschnitt \in[headers]) verwendet
  werden soll. Die Gefahr ist, dass der aktuelle Namensraum durch eine
  Unmenge von Bezeichnern überflutet wird und man nicht genau ermitteln
  kann, woher ein spezieller Bezeichner stammt.

  Wichtig ist, dass zwar alle Namen aus dem angegebenen Namensraum
  zugreifbar sind, diese aber nicht im aktuellen Bereich als deklariert
  gelten:

  \startcpp
  // namespace_directive.cpp
  #include <iostream>

  int main() {
      using namespace std;

      int cin{0};
      cout << cin << endl;
  }
  \stopcpp

  Als Ausgabe wird erwartungsgemäß nur \type{0} erscheinen.
\item
  Weiters kommt \type{using} in Form eines Typalias ({\em type alias
  declaration}) vor, der einen neuen Namen im bestehenden Bereich für
  einen bestehenden Typ deklariert. Es wird kein neuer Typ erzeugt und
  es kann auch kein bestehender Typ verändert werden. Es wird ein neuer
  Name für einen schon bestehenden Typ zum lokalen Scope hinzugefügt.

  Das macht manchmal Sinn, wenn der ursprüngliche Typname zu kompliziert
  oder zu lange ist. Im folgenden Beispiel ist dies zu sehen:

  \startcpp
  // using.cpp
  #include <iostream>
  #include <vector>

  int main() {
      using IntStack = std::vector<int>;

      IntStack stack{};

      stack.push_back(1);
      stack.push_back(2);
      stack.push_back(3);

      std::cout << "top: " << stack.back() << std::endl;
      stack.pop_back();
      std::cout << stack.back() << std::endl;
      stack.pop_back();
      std::cout << "bottom: " << stack.back() << std::endl;
      stack.pop_back();
  }
  \stopcpp

  Es kommt erwartungsgemäß zu folgender Ausgabe:

  \startcpp
  top: 3
  2
  bottom: 1
  \stopcpp

  Die Funktionsweise ist folgende:

  \startitemize[packed]
  \item
    Hier wird dem Typ \type{std::vector<int>} der neue Name
    \type{IntStack} gegeben, der in weiterer Folge zur Definition der
    Variable \type{stack} verwendet wird.
  \item
    Mittels \type{push_back()} kann man einem Vektor hinten Elemente
    anfügen.
  \item
    Der Zugriff auf das letzte Element erfolgt mittels \type{back()}.
  \item
    Das letzte Element kann mittels \type{pop_back()} aus dem Vektor
    entfernt werden. Diese Methode liefert keinen Wert zurück.
  \item
    Beachten muss man, dass man jetzt auf \type{cout} und \type{endl}
    explizit mittels dem \type{::} Operator zugreifen muss, da keine
    \type{using namespace std;} Direktive im Programm enthalten ist.
  \stopitemize

  Beachte, dass es sich um eine Deklaration handelt und hiermit das
  folgende Programm der Compiler {\em nicht} übersetzen wird, da es zwei
  Deklarationen mit dem gleichen Bezeichner enthält:

  \startcpp
  // using_wrong.cpp
  #include <iostream>
  #include <vector>

  int main() {
      using IntStack = std::vector<int>;

      using IntStack = std::vector<char>;
  }
  \stopcpp
\item
  In der Form einer \type{using}-Deklaration wird diese verwendet, um
  gezielt einen Bezeichner aus einem Namensraum im aktuellen
  Geltungsbereich verwenden zu können. Auch hierbei wird ein Bezeichner
  zum lokalen Scope hinzugefügt.

  \startitemize
  \item
    Einerseits handelt es sich um eine Kurzschreibweise eines Typalias.
    Die folgende \type{using}-Deklaration:

    \startcpp
    using std::vector;
    \stopcpp

    ist äquivalent zu dem folgenden Typalias:

    \startcpp
    using vector = std::vector;
    \stopcpp
  \item
    Andererseits kann eine \type{using}-Deklaration auch verwendet
    werden, um einen Bezeichner, der keinen Typ darstellt, im aktuellen
    Bereich verwenden zu können:

    Im gerade besprochenen Programm, das den Typalias demonstriert hat,
    haben wir gesehen, dass wir jetzt explizit \type{std::cout} und
    \type{std::endl} verwenden mussten. Da es sich allerdings um derart
    häufig verwendete Objekte handelt, macht es Sinn, diese gezielt im
    aktuellen Geltungsbereich zu importieren.

    Füge deshalb in deiner Datei \type{using.cpp} {\em vor} dem Beginn
    von \type{main} die folgenden beiden Zeilen ein:

    \startcpp
    using std::cout;
    using std::endl;
    \stopcpp

    Damit kannst du in weiterer Folge die Objekte \type{cout} und
    \type{endl} wieder unqualifiziert verwenden. Beachte den kleinen
    Unterschied, der hier absichtlich eingebaut wurde: Die beiden
    \type{using}-Deklarationen sind in der gesamten Datei verfügbar,
    während der Typalias \type{IntStack} nur in der Funktion \type{main}
    zur Verfügung steht.
  \stopitemize
\stopitemize

\stopcomponent
