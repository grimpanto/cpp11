
\startcomponent comp_templates
\product prod_book

\chapter{Templates}
\section[templatefoundations]{Generelles zu Templates}

Deklaration und Definition

\section[functiontemplates]{Funktionstemplates}

\cpp bietet einen sehr mächtigen Mechanismus an, um generische
Funktionen und auch Klassen zu entwickeln. In diesem Abschnitt werden
wir uns ansehen, wie wir generische Funktionen entwickeln können.

Nehmen wir nochmals unsere Funktion \type{swap} aus dem Abschnitt
\in[rvalueref] an:

\startcpp
void swap(string& a, string& b) {
    string tmp{move(a)};
    a = move(b);
    b = move(tmp);
}
\stopcpp

Was ist jetzt wenn, wir eine weitere Funktion wollen, die genau das
Gleiche erledigt, nur für ganze Zahlen. Natürlich können wir einfach die
obige Funktion kopieren und jedes Vorkommen von \type{string} durch
\type{int} ersetzen. Damit haben wir eine überladene Funktion
geschrieben, die genau dies erledigt. Jetzt könnte es sein, dass wir
noch so eine Funktion wollen, die dies für Gleitkommazahlen erledigt.
Natürlich können wir wieder mit dem Kopieren und Abändern beginnen und
es würde auch funktionieren, aber es macht keinen Sinn! Das kann der
Compiler viel besser als wir.

Deshalb wollen wir Schablonen (engl. {\em templates} oder eingedeutscht
Templates) einsetzen:

\startcpp
// swaptemplate.cpp
#include <iostream>
#include <memory>

template <typename T>
void swap(T& a, T& b) {
    T tmp{std::move(a)};
    a = std::move(b);
    b = std::move(tmp);
}


int main() {
    int i1{1};
    int i2{2};

    swap(i1, i2);
    std::cout << "i1 = " << i1 << ", i2 = " << i2 << std::endl;
}
\stopcpp

Die Ausgabe wird genau das Ergebnis der Funktion \type{swap}
widerspiegeln. XXX

Variable Anzahl an Parameter

\subsection[genericlambdatemplates]{Generische Lambdafunktion in
Templates}

\startannotation{\cppXIV}

Generische Lambdafunktionen (siehe Abschntt \in[lambdafunc]) können im
Zusammenhang mit Templates sinnvoll eingesetzt werden:

XXX ist abzuändern und fertigzustellen

\starttyping
template<class C>
void print_elements(const C & c) {
    std::for_each(begin(c), end(c), [](auto & x) {
        std::cout << x << " ";
    });
    std::cout << std::endl;
}
\stoptyping

\starttyping
vector<int> v{ 1, 2, 3 };
list<string> w{ "C++", "is", "cool" };

print_elements(v);
print_elements(w);
\stoptyping

\stopannotation

\section[variablentemplates]{Variablentemplates}

\cppXI erlaubt Templates nur für Funktionen, Klassen und Typaliase und
\cppXIV
erweitert die Verwendung von Templates auf Variablen:

\startannotation{\cppXIV}

\starttyping
// vartemplates.cpp
#include <iostream>

using namespace std;

template <typename T>
constexpr T pi = T{3.1415926535897932385};

template <typename T>
T circle_perimeter(T r) {
    return 2 * r * pi<T>;
}

int main() {
    cout << circle_perimeter(2.5) << endl;
}
\stoptyping

\stopannotation

\stopcomponent
