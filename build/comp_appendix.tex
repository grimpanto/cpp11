
\startcomponent comp_appendix
\product prod_book

\chapter{Anhang}
\section[devtools]{Entwicklungswerkzeuge}

Um mit \cpp entwickeln zu können, werden Entwicklungswerkzeuge wie
Compiler, Linker und Debugger benötigt. Diese sind naturgemäß je
Betriebssystem unterschiedlich. Prinzipiell steht die Gnu Compiler
Collection sowohl für Windows als auch für Mac OSX und Linux zur
Verfügung. Trotzdem macht es durchaus Sinn, eine auf die jeweilige
Betriebssystemplattform abgestimmte Variante zu wählen:

\startitemize
\item
  Unter {\em Windows} wird meist der Compiler Microsoft Visual \cpp
  verwendet, der in der Entwicklungsumgebung Microsoft Visual Studio
  enthalten ist. Es wird von Microsoft auch in einer kostenlosen Version
  angeboten.
\item
  Unter {\em Mac OSX} muss die Entwicklungsumgebung XCode von der
  Entwicklersite von Apple besorgt und installiert werden. Bei der
  Installation ist zu beachten, dass die {\em command line tools} zu
  installieren sind. Die aktuelle Version enthält den \cpp Compiler
  Clang.
\item
  Unter Linux sind die Entwicklungswerkzeuge ebenfalls getrennt zu
  installieren. Hier wird meist der \cpp Compiler \type{g++} der Gnu
  Compiler Collection verwendet.

  Für Ubuntu sieht der Befehl zur Installation folgendermaßen aus:

  \startsh
  sudo apt-get install build-essential libgl1-mesa-dev
  \stopsh

  Verwendest du eine andere oder nicht kompatible Distribution, dann
  wende dich bitte and die entsprechende Dokumentation.

  Willst du unter Linux mit \cppXIV programmieren, dann benötigst du
  \type{g++} mindestens in der Version 4.9, falls diese in deiner
  Distribution angeboten wird oder du installierst den \cpp Compiler
  Clang. Das kann in Ubuntu folgendermaßen realisiert werden:

  \startsh
  sudo apt-get install clang-3.4 libc++-dev binutils
  \stopsh
\stopitemize

\section[compilation]{Übersetzung eines \cpp Programmes}

Hier geht es nur darum, die Installation zu testen und zu zeigen, wie
ein \cpp Programm kompiliert wird. Natürlich kann zum Schreiben als auch
zum Übersetzen und zum Ausführen eine Entwicklungsumgebung verwendet
werden, aber werden wir hier eine Shell benützen.

\startdrawer

Eine Shell (zu Deutsch: Schale) ist ein Programm, das wie eine Schale
sich um das Betriebssystem legt und dem Benutzer eine Schnittstelle
bietet, um mit dem Betriebssystem per Kommandos zu kommunizieren.

Diese Kommandos werden mittels textbasierter Befehle eingegeben und die
Antworten des Betriebssystem werden als Text wieder angezeigt.

Solch eine Shell wird oft auch als Kommandozeileninterpreter bezeichnet,
weil es eben ein Programm ist, der Kommandos zeilenweise interpretiert,
ausführt und die Ergebnisse zeilenweise anzeigt.

Je Betriebssystem gibt es verschiedene derartige Programme.
\stopdrawer

Erstelle das klassische \quotation{Hello World} Programm mit einem
Texteditor deiner Wahl. Im Abschnitt \in[helloworld] auf der Seite
\at[helloworld] wird der gesamte Quellcode detailliert erklärt:

\startcpp
#include <iostream>

int main() {
    std::cout << "Hello, world!" << std::endl;
}
\stopcpp

Speichere diesen Quelltext in eine Datei \type{hello.cpp}.

Da es sich bei \cpp um eine \quotation{compiled language} handelt, muss
der Sourcecode zuerst noch in eine ausführbare Datei übersetzt werden,
bevor das Programm verwendet werden kann.

Jetzt geht es darum diese Quelldatei in eine ausführbare Datei zu
übersetzen und mit den notwendigen Bibliotheken zu linken.

\subsection[windows]{Windows}

Öffne eine Shell! Unter Windows kannst du \type{cmd} oder die PowerShell
verwenden.

Wird Visual Studio verwendet, dann sollte der Befehl

\startsh
cl /clr hello.cpp
\stopsh

in der Shell eingegeben werden können. Damit wird eine ausführbare Datei
mit dem Namen \type{hello.exe} erzeugt, die einfach durch Eingabe von
\type{hello} gestartet werden kann.

Um Warnungen zu sehen, empfehle ich die Option \type{/Wall} mitzugeben.

Will man eine einzelne Datei lediglich in seine Objektdatei übersetzen,
dann ist die Option \type{/c} (für \quotation{compile}) zu verwenden.
Die entstehende Datei heißt \type{hello.obj} (für Objektdatei).

Unter Umständen müssen mehrere Dateien angegeben werden, dann sind diese
einfach in der Kommandozeile hintereinander anzugeben.

\subsection[mac-osx]{Mac OSX}

Öffne eine Shell! Verwende unter Mac OSX die Applikation
\type{Terminal}, in das die eigentliche Shell läuft.

Aktuelle XCode Versionen verwenden den \cpp Compiler clang. Gib den
folgenden Befehl in der Shell ein:

\startsh
clang++ -stdlib=libc++ -std=c++11 -lc++abi -o hello hello.cpp
\stopsh

Die Option \type{-o} (engl. {\em output}) bewirkt, dass das lauffähige
Programm mit dem Namen \type{hello} erstellt wird. \type{-std=c++11}
gibt an, dass die Sprachversion \cppXI verwendet werden soll. Die
restlichen Parameter geben die zu verwendete Bibliothek an.

Anstatt \type{-std=c++11} kann auch \type{-std=c++1y} verwendet werden,
wenn eine aktuelle Version von \type{clang} installiert ist. Dies
ermöglicht dann auch Features von \cppXIV zu verwenden.

Auch \type{clang} kennt die Option \type{-Wall}, um Warnungen
anzuzeigen.

Für die weiteren Beispiele gehe ich davon aus, dass die
Umgebungsvariable \type{PATH} so gesetzt ist, dass das aktuelle
Verzeichnis enthalten ist. D.h., dass der Punkt \type{.} enthalten ist.

Jetzt kann durch Eingabe von \type{hello} einfach das Programm gestartet
werden.

Will man eine einzelne Datei lediglich in seine Objektdatei übersetzen,
dann ist die Option \type{-c} (für \quotation{compile}) anzuhängen und
die Option \type{-o hello} nicht anzugeben. Die entstehende Datei heißt
\type{hello.o} (für Objektdatei).

Unter Umständen müssen mehrere Dateien angegeben werden, dann sind diese
einfach in der Kommandozeile hintereinander anzugeben.

\subsection[linux]{Linux}

Öffne eine Shell! Unter Linux kannst du eines der vielen verschiedenen
Terminalprogramme verwenden. Unter Linux läuft innerhalb des
Terminalprogrammes die eigentliche Shell.

Für die weiteren Beispiele gehe ich davon aus, dass die
Umgebungsvariable \type{PATH} so gesetzt ist, dass das aktuelle
Verzeichnis enthalten ist. D.h., dass der Punkt \type{.} enthalten ist.

Unter Linux sieht der Befehl folgendermaßen aus:

\startsh
g++ hello.cpp -std=c++11 -o hello
\stopsh

Die Option \type{-o} (engl. {\em output}) bewirkt, dass das lauffähige
Programm mit dem Namen \type{hello} erstellt wird. \type{-std=c++11}
gibt an, dass die Sprachversion \cppXI verwendet werden soll.

Anstatt \type{-std=c++11} kann auch \type{-std=c++14} oder
\type{-std=c++1y} verwendet werden, wenn eine aktuelle Version von
\type{g++} installiert ist. Dies ermöglicht dann auch Features von
\cppXIV zu verwenden.

Will man sich Warnungen anzeigen lassen, dann empfehle ich die Optionen
\type{-Wconversion} und \type{-Wall} anzugeben. \type{g++} hat sehr
viele Optionen und diese sollten je Bedarf ausgesucht werden.

Für die weiteren Beispiele gehe ich davon aus, dass die
Umgebungsvariable \type{PATH} so gesetzt ist, dass das aktuelle
Verzeichnis enthalten ist. D.h., dass der Punkt \type{.} enthalten ist.

Jetzt kann durch Eingabe von \type{hello} einfach das Programm gestartet
werden.

Will man eine einzelne Datei lediglich in seine Objektdatei übersetzen,
dann ist die Option \type{-c} (für \quotation{compile}) anzuhängen und
die Option \type{-o hello} nicht anzugeben. Die entstehende Datei heißt
\type{hello.o} (für Objektdatei).

Unter Umständen müssen mehrere Dateien angegeben werden, dann sind diese
einfach in der Kommandozeile hintereinander anzugeben.

\section[access_exit_code]{Zugreifen auf den Exit-Code eines Programmes}

Will man auf den Exit-Code eines Programmes zugreifen und weiß nicht wie
dann hilft dieser Abschnitt weiter.

Schreibe zuerst den folgenden Code in eine Datei \type{minimal.cpp} und
übersetze dieses Programm in eine ausführbare Datei \type{minimal}:

\startcpp
int main() {
    return 0;
}
\stopcpp

Im folgenden wird erklärt wie auf den Rückgabewert je verwendetem
Betriebssystem zugegriffen werden kann.

\subsection[windows-1]{Windows}

Unter Windows gibt es je nach verwendeter Shell verschiedene
Möglichkeiten. Wird eine \type{cmd} Shell verwendet, dann kommt man mit
folgendem Befehl ans Ziel:

\startsh
minimum
echo %ERRORLEVEL%
0
\stopsh

Danach wird in unserem konkreten Fall der Wert \type{0} angezeigt.

Wird die PowerShell verwendet, dann sieht der korrekte Befehl
folgendermaßen aus:

\startsh
minimum
echo $LastExitCode
0
\stopsh

\subsection[mac-osx-und-linux]{Mac OSX und Linux}

Sowohl für Mac OSX als auch für Linux gibt es eine Vielzahl von
verschiedenen Shells. Normalerweise kommt eine Shell zum Einsatz, die
\type{bash} heißt. Viele Shells sind bezüglich der Ermittlung des
Rückgabewertes kompatibel zu dieser Shell.

Unter Mac OSX und Linux kann man meist auf der Kommandozeile
folgendermaßen auf diesen Rückgabewert zugreifen:

\startsh
minimum
echo $?
0
\stopsh

Auch hier wird mit unserem Beispiel natürlich wieder \type{0} angezeigt
werden.

Verwendest du die sehr gute Shell \type{fish}, dann sieht der Befehl
folgendermaßen aus:

\startsh
minimum
echo $status
0
\stopsh

Solltest du eine andere Shell verwenden, die sich diesbezüglich wiederum
anders verhält, dann bist du wahrscheinlich ein Spezialist und benötigst
wahrscheinlich keine weitere Hilfe.

\section[literaturhinweise]{Literaturhinweise}

Hier möchte ich sinnvolle Ergänzungen für das weitere Studium in
\cpp anführen.

Die folgendenden Bücher gelten als \quotation{Standard}:

\startitemize[packed]
\item
  Die \quotation{Bibel} ist sicher das Buch des Erfinders Bjarne
  Stroustrup von \cpp{} \cite[stroustrup2013].
\item
  Als das Standardwerk zur Standardbibliothek gilt das Buch
  \cite[josuttis2012] von Nicolai Josuttis.
\item
  Als Online-Quelle hauptsächlich zur Standardbibliothek empfehle ich
  \cite[cppreference].
\stopitemize

Bist du mehr an einer Einführung interessiert, dann sind die folgenden
Werke unter Umständen für dich interessant:

\startitemize[packed]
\item
  Ist man mehr an einer eher deutschsprachigen, umfassenden Einführung
  interessiert, die sich auch an Nicht-Programmierer richtet, dann ist
  \cite[wolf2014] zu empfehlen.
\item
  Ein ebenfalls gutes deutschsprachiges und ebenfalls umfangreiches
  Buch, das viele praktische Algorithmen enthält ist
  \cite[breymann2011].
\item
  Will man eher ein englischsprachige, gute Einführung, dann ist
  \cite[deitel2014] zu empfehlen.
\stopitemize

Hast du schon Erfahrungen mit \cpp und möchtest speziell die neuen
Features von \cppXI kennenlernen oder möchtest diese kompakt in einem
Buch nachlesen können,

\startitemize[packed]
\item
  dann empfehle ich das Buch \cite[will2012]!
\stopitemize

Zu guter letzt kann ich noch zwei ausgezeichnete Bücher empfehlen, die
jedoch eindeutig schon in die Kategorie weiterführende Literatur
einzuordnen sind:

\startitemize[packed]
\item
  Interessiert man sich für parallele Anwendungen, dann empfehle ich
  \cite[williams2012].
\item
  Ist man eher daran interessiert, wie man gute, wiederverwendbare
  Programme entwickelt, dann ist \cite[reddy2011] sehr zu empfehlen.
\stopitemize

\stopcomponent
