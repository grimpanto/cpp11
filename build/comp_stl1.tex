
\startcomponent comp_stl1
\product prod_book

\chapter{Einführung in die Standardbibliothek}
\section[überblick]{Überblick}

Die Standardbibliothek von \cpp ist sehr umfangreich! In diesem
Abschnitt werden wir einen Überblick bekommen und die wesentlichen
Prinzipien verstehen werden.

Grob kann man die Standardbibliothek in verschiedene Kategorien
gliedern. Hier folgen die Kategorien gruppiert im Überblick:

\subsection[sprachunterstützung]{Sprachunterstützung,}

In diese Kategorie fallen Header, die direkt als Ergänzung der
Programmiersprache \cpp zuzuordnen sind. Wichtige Headerdateien sind:

\startitemize
\item
  \type{<limits>}: Diese Headerdatei haben wir schon kennengelernt, da
  diese uns wichtige Informationen über die Implementierung der
  arithmetischen Datentypen liefert.
\item
  \type{<cstdint>}: Auch diese Headerdatei kennen wir bereits. Diese
  stellt weitere Typen für ganze Zahlen mit definierter Größe zur
  Verfügung.
\item
  \type{<typeinfo>}: In \type{<typeinfo>} ist eine gewisse Unterstützung
  enthalten, um zur Laufzeit Information über die Typen von
  Speicherobjekten zu ermitteln.
\item
  \type{<exception>}: Hier sind Funktionen und Typen enthalten, die für
  das Excpetion-Handling nützlich und notwendig sind.
\item
  \type{<initializer_list>}: Diese Headerdatei kennen wir ebenfalls
  schon: Sie stellt den Typ \type{initializer_list} zur Verfügung.
\stopitemize

\subsection[utility]{Utility}

Diese Kategorie enthält sinnvolle Funktionen, die oft benötigt werden
und allgemeiner Natur sind. Hier gibt es die folgenden wichtigen
Headerdateien:

\startitemize[packed]
\item
  \type{<utility>}
\item
  \type{<tuple>}
\item
  \type{<bitset>}
\item
  \type{<memory>}
\item
  \type{<functional>}
\item
  \type{<ration>}
\item
  \type{<chrono>} und \type{<ctime>}
\stopitemize

\subsection[mehrsprachigkeit]{Mehrsprachigkeit}

xxx

\subsection[strings]{Strings}

xxx

\subsection[ein--und-ausgabe]{Ein- und Ausgabe}

xxx

\subsection[container-algorithmen-iterator]{Container, Algorithmen,
Iterator}

xxx

\subsection[numerik]{Numerik}

xxx

\subsection[reguläre-ausdrücke]{Reguläre Ausdrücke}

xxx

\subsection[unterstützung-für-thread-basierte-nebenläufige-anwendungen]{Unterstützung
für Thread-basierte, nebenläufige Anwendungen}

xxx

\section[sortieren]{Sortieren}

Schauen wir uns folgendes Beispiel an, das es erlaubt beliebige
Sequenzen zu sortieren, ohne \type{begin} und \type{end} verwenden zu
müssen:

\startcpp
// sorting.cpp
#include <iostream>
#include <algorithm>

namespace Estd {
    using namespace std;

    template <typename T>
    void sort(T& seq) {
        std::sort(begin(seq), end(seq));
    }
}

using namespace Estd;

int main() {
    int nums[]{3, 5, 2, 1, 4};

    sort(nums);

    for (auto n : nums) {
        cout << n << ' ';
    }
    cout << endl;
}
\stopcpp

\section[vector]{vector}

Die schon bekannte Möglichkeit ist, eine Instanz von \type{vector} zu
verwenden, wie wir diese schon kennengelernt haben. Ein \type{vector}
unterscheidet sich von einem rohen Array dadurch, dass es in der Größe
veränderbar ist.

Wir wollen wir uns jetzt die Funktionsweise und wichtige Funktionen der
Klasse genauer ansehen. Ein \type{vector} hat eine aktuelle Größe und
eine Kapazität, die die Größe des reservierten Speicherbereiches angibt:

\startcpp
#include <iostream>
#include <vector>

using namespace std;

int main() {
    vector<int> v{1, 2};
    cout << "size: " << v.size() << endl;
    cout << "capacity: " << v.capacity() << endl;
}
\stopcpp

Die Ausgabe könnte folgendermaßen aussehen:

\startsh
size: 2
capacity: 2
\stopsh

Da zwei Elemente, nämlich die Zahlen 1 und 2, im Vektor enthalten sind,
ist die aktuelle Größe 2. Weiters ist die Kapazität ebenfalls 2.

Hängen wir in weiterer Folge diese Codezeilen an:

\startcpp
v.push_back(3);

cout << "push_back(3)" << endl;
cout << "size: " << v.size() << endl;
cout << "capacity: " << v.capacity() << endl;
\stopcpp

Die Ausgabe sieht bei mir jetzt folgendermaßen aus:

\startsh
size: 2
capacity: 2
push_back(3)
size: 3
capacity: 4
\stopsh

Damit sind im Moment die Zahlen 1, 2 und 3 im Vektor enthalten und
hiermit beträgt die aktuelle Gräße ebenfalls 3. Nach der Initialisierung
mit den beiden Werten 1 und 2 hat der Vektor \type{v} nur einen
Speicherbereich vom Betriebssystem angefordert, der in der Lage war 2
Elemente aufzunehmen. Dann haben wir ein weiteres Element hinten
angehängt und hiermit hat der Vektor einen neuen Speicherbereich
anfordern müssen. Offensichtlich hat der Vektor den Speicherbereich
doppelt so groß angefordert.

Testen wir weiter durch Anfügen der folgenden Codezeilen:

\startcpp
v.push_back(4);
v.push_back(5);
cout << "push_back(4)" << endl;
cout << "push_back(4)" << endl;
cout << "size: " << v.size() << endl;
cout << "capacity: " << v.capacity() << endl;
\stopcpp

Die zusätzliche Ausgabe sieht bei mir folgendermaßen aus:

\startsh
push_back(4)
push_back(4)
size: 5
capacity: 8
\stopsh

Hier sehen wir sehr deutlich, dass der Vektor in diesem Schritt wieder
einen doppelt so großen Speicherbereich angefordert hat. Prinzipiell
hängt die Vorgehensweise von der Implementierung der Standardbibliothek
durch den Hersteller zusammen, aber diese Art ist sehr sinnvoll. Wird
ein neues Element zum Vektor hinzugefügt, das in den bestehenden
Speicherbereich nicht abgelegt werden kann, dann muss der Vektor einen
neuen Speicherbereich vom Betriebssystem anfordern, die alten Inhalte in
den neuen Speicherbereich kopieren und den alten Speicherbereich dem
Betriebssystem zurückgeben.

Damit dieser Vorgang nicht bei jedem Anfügen erfolgen muss, erhöht der
Vektor die zusätzliche Größe in der konkreten auf die doppelte Größe.

\stopcomponent
