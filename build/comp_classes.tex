
\startcomponent comp_classes
\product prod_book

\chapter{Klassen}
Die grundlegende Idee der Objektorientierung ist, dass Daten und
zugehörige Operationen in einer Einheit gruppiert werden. Dazu steht als
Hilfsmittel das bekannte Konzept der {\em Klasse} zur Verfügung, das den
Aufbau und das Verhalten gleichartiger Objekte beschreibt.

\cpp bietet vielfältige Möglichkeiten, um Klassen zu beschreiben.

\section[classdeclaration]{Deklaration und Definiton einer Klasse}

In diesem Abschnitt werden wir uns mit den grundlegenden Möglichkeiten
beschäftigen, wie man in \cpp Klassen deklarieren und definieren kann.

Wir haben schon gesehen, dass mittels \type{struct} eine Klasse
definiert wird, deren Attribute und Methoden alle öffentlich zur
Verfügung stehen.

Ein Attribut wird in \cpp als {\em data member} (dt. so viel wie
Datenmitglied) und eine Methode wird als {\em member function} (dt. so
viel wie Mitgliedsfunktion) bezeichnet.

Schauen wir uns zuerst ein Beispiel an, das eine Person definiert, die
vorerst nur über zwei Attribute verfügt:

\startcpp
// class.cpp
#include <iostream>

using namespace std;

struct Person {
    string name;
    double weight;
};

int main() {
    Person p;
    p.name = "Max";
    p.weight = 79.5;
    Person q;
    cout << "Name: " << p.name << ", Gewicht: "
         << p.weight << endl;
    cout << "Name: " << q.name << ", Gewicht: "
         << q.weight << endl;
}
\stopcpp

Die Ausgabe dieses Programmes könnte folgendermaßen aussehen:

\startsh
Name: Max, Gewicht: 79.5
Name: , Gewicht: 2.18846e-314
\stopsh

Aus diesem Quelltext und der Ausgabe lässt sich einiges ablesen:

\startitemize
\item
  Wie schon gesagt, stehen alle Attribute eines \type{struct}s
  öffentlich zur Verfügung. Das bedeutet, dass von außerhalb der Klasse
  darauf zugegriffen werden kann, wie dies beim Zugriff auf die
  Attribute \type{name} und \type{weight} in der Funktion \type{main} zu
  sehen ist.
\item
  Da es sich bei \type{p} und \type{q} um Objekte des Typs \type{Person}
  handelt, haben beide unterschiedliche, eigene Werte für die Attribute.
\item
  Weiters erkennen wir, dass weder \type{name} noch \type{weight} beim
  Anlegen initialisiert werden. Lediglich im Fall von \type{p} werden
  den beiden Attribute nachträglich Werte zugewiesen. Bei der Ausgabe
  der uninitialisierten Attribute von \type{q} sehen wir, dass bei der
  Ausgabe von \type{name} ein Leerstring ausgegeben wird, während bei
  der Ausgabe der Größe ein anscheinend beliebiger Wert aufscheint.

  Daraus können wir ablesen, dass Attribute eines benutzerdefinierten
  Typs initialisiert werden, Attribute eines eingebauten Typs jedoch
  nicht!
\stopitemize

Man kann allerdings auch direkt in der Deklaration der Klasse die
Attribute initialisieren. Ändere dazu die Deklaration des Attributs
\type{weight} wie folgt ab, sodass das Gewicht mit \type{0.0}
initialisiert wird:

\startcpp
double weight{};
\stopcpp

Danach wird die Ausgabe folgendermaßen aussehen:

\startcpp
Name: Max, Gewicht: 79.5
Name: , Gewicht: 0
\stopcpp

Will man die den Namen und das Gewicht beim Anlegen der Person angeben
und somit die nachträgliche Zuweisung vermeiden, dann benötigt man einen
Konstruktor:

\startcpp
struct Person {
    Person(string name, double weight) {
        this->name = name;
        this->weight = weight;
    }
    string name;
    double weight{};
};
\stopcpp

Das Anlegen der Instanzen soll jetzt folgendermaßen aussehen:

\startcpp
Person p{"Max", 79.5};
Person q("", 0);
\stopcpp

Damit funktioniert das Programm jetzt wieder wie gewohnt und wir können
uns den Details zuwenden:

\startitemize
\item
  In \cpp ist ein Konstruktor also eine Methode, die den Namen der
  Klasse trägt und keinen Rückgabetyp aufweist. Dieser Konstruktor wird
  ausgeführt, wenn ein Objekt dieser Klasse angelegt wird. Daher wird
  ein Konstruktor zum Initializieren des Objektes verwendet.
\item
  Der Rumpf des Konstruktors wurde direkt innerhalb der
  Klassendeklaration angeführt. Das hat die gleiche Bedeutung als wenn
  wir eine Funktion mit dem Spezifizierungssymbol \type{inline} markiert
  hätten. Wir weisen den Compiler an, den Rumpf dieses Konstruktors wenn
  möglich direkt in den Code einzufügen.
\item
  Innerhalb des Rumpfes sehen wir, dass wir einen Zeiger \type{this}
  verwenden. Jede Methode eines Objektes verfügt implizit über so einen
  Zeiger, der direkt auf das Objekt selber verweist. In diesem konkreten
  Fall wurde diese Konstruktion verwendet, um die Namen der Parameter
  gleich benennen zu können wie die Namen der Instanzvariablen.

  Innerhalb des Rumpfes wurde mit der Angabe der Parameter, die
  Bezeichner dieser Parameter in den Gültigkeitsbereich des Rumpfes des
  Konstruktors übernommen. Mittels \type{this->} steht eine Möglichkeit
  zur Verfügung, auf die Instanzvariablen zuzugreifen.

  Ohne Verwendung von \type{this} funktioniert unser Beispiel nicht wie
  wir uns dies erwarten:

  \startcpp
  Person(string name, double weight) {
      name = name;
      weight = weight;
  }
  \stopcpp

  Der Compiler wird das Beispiel ohne Fehler übersetzen, aber es wird
  verständlicherweise nicht wie erwartet funktionieren, da die
  Parameterdeklaration die Deklaration der Instanzvariablen
  überschattet. So erhält man in diesem Fall lediglich, dass die
  Parameter sich selbst die eigenen Werte zuweisen. Die Instanzvariablen
  werden nicht gesetzt.

  Alternativ hätten wir den auch Konstruktor folgendermaßen definieren
  können:

  \startcpp
  Person(string name_, double weight_) {
      name = name_;
      weight = weight_;
  }
  \stopcpp

  Damit lauten die Bezeichner der Parameter anders als die Bezeichner
  der Instanzvariable. Aus diesem Grund können die Bezeichner der
  Instanzvariablen ohne \type{this->} verwendet werden. Das hat
  allerdings rein syntaxtische Auswirkungen, da der Compiler -- von der
  Bedeutung her -- immer ein \type{this->} voranstellt:

  \startcpp
  Person(string name_, double weight_) {
      this->name = name_;
      this->weight = weight_;
  }
  \stopcpp
\item
  Die angegebenen Parameter werden beim Anlegen eines Objektes
  mitgegeben. Wir sehen, dass wir die Initialisierung sowohl in der Form
  der vereinheitlichten Initialisierung als auch in der Form eines
  Funktionsaufrufes vornehmen können.

  Wie schon besprochen haben, ist es sinnvoller die vereinheitlichte
  Form zu wählen.
\item
  Weiters sehen wir, dass wir jetzt auch die Variable \type{q} mit Hilfe
  dieses Konstruktors initialisieren müssen. Das lässt sich leicht
  verifizieren, indem die ursprüngliche Deklaration in der Form von
  \type{Person q;} wieder eingesetzt wird. Der Compiler wird dies
  zurückweisen, da kein Konstruktor zur Verfügung steht, der keine
  Parameter erwartet.

  Jetzt stellt sich die Frage, wie dies ohne Konstruktor funktionieren
  konnte. Das liegt daran, dass der Compiler einen Default-Konstruktor
  generiert, wenn es keinen anderen Konstruktor gibt.

  Ein generierter Default-Konstruktor hat keine Parameter und
  initialisiert die Instanzvariablen gemäß den Regeln, die wir schon in
  diesem Abschnitt kennengelernt haben.

  Natürlich steht es uns frei einen Default-Konstruktor selber zu
  schreiben und damit \type{Person q} wieder wie gewohnt verwenden zu
  können:

  \startcpp
  Person() {}
  \stopcpp

  In unserem konkreten Fall geht dies in dieser einfachen Art und Weise,
  da wir direkt bei der Deklaration der Instanzvariablen die gewünschte
  Initialisierung vorgenommen haben.

  Alternativ könnte man Default-Parameter zu dem Konstruktor mit den
  Parametern hinzufügen, wodurch ebenfalls ein Default-Konstruktor
  entsteht, da dieser Konstruktor jetzt

  \startcpp
  Person(string name="", double weight=0) {
      this->name = name;
      this->weight = weight;
  }
  \stopcpp

  Damit besteht die Möglichkeit, die Initialisierungen direkt bei der
  Instanzvariable \type{weight} zu entfernen:

  \startcpp
  double weight;
  \stopcpp

  Allerdings behebt dies nicht das Problem, dass die Variable
  \type{name} zuerst mit einem Leerstring initialisiert wird und danach
  mit dem übergebenen Parameter im Konstruktor überschrieben wird (durch
  eine Zuweisung). Diesen Umstand kann mit folgender Syntax zu Leibe
  gerückt werden:

  \startcpp
  Person(string name="", double weight=0) : name{name},
    weight{weight} { }
  \stopcpp

  Damit wird die Instanzvariable \type{name} mit dem Parameter
  \type{name} und die Instanzvariable \type{weight} mit dem Parameter
  \type{weight} initialisiert. Eine weitere Initialisierung \type{name}
  beziehungsweise \type{weight} findet nicht statt.
\stopitemize

Eine Objekt kann auch default-mäßig mittels einer Kopie eines Objektes
gleichen Typs initialisiert werden. Dazu stellt der Compiler diese
Funktionalität selbständig in der Art zur Verfügung, dass jede
Instanzvariable kopiert wird:

\startcpp
Person r{p};
cout << "Name: " << r.name << ", Gewicht: " << r.weight << endl;
\stopcpp

Die Ausgabe wird wie erwartet wie die Ausgabe von \type{p} sein.

Auf die gleiche Art und Weise besteht default-mäßig die Möglichkeit ein
Objekt einem anderen Objekt zuzuweisen. Auch hierfür stellt der Compiler
die Funktionalität zur Verfügung, sodass jede Instanzvariable kopiert
wird:

\startcpp
r = q;
cout << "Name: " << r.name << ", Gewicht: " << r.weight << endl;
\stopcpp

Damit ergibt sich die Ausgabe wie bei der Ausgabe von \type{q}.

Im Sinne der Datenkapselung (engl. {\em encapsulation}) macht es keinen
Sinn, alle Attribute und Methoden öffentlich verfügbar zu machen.

Ändere deshalb das \type{struct} wie folgt ab:

\startcpp
struct Person {
    Person(string name="", double weight=0) : name{name},
      weight{weight} { }
  private:
    string name;
    double weight;
};
\stopcpp

In dieser Form lässt sich das Programm jedoch nicht mehr übersetzen! Das
Hinzufügen von \type{private:} bedeutet, dass die folgenden Bezeichner
nicht mehr von außerhalb der Klasse aus zugreifbar sind. Damit kann
\type{p.name} nicht mehr funktionieren.

Wie ist vorzugehen? Jetzt ist es Zeit, Methoden einzuführen, die
öffentlich zugreifbar sind und die jeweilige Instanzvariable
zurückliefern. Solche Methoden werden oft als \quotation{getter method}
bezeichnet:

\startcpp
struct Person {
    Person(string name="", double weight=0) : name{name},
      weight{weight} { }
    string get_name() { return name; }
    double get_weight() { return weight; }
  private:
    string name;
    double weight;
};
\stopcpp

Natürlich muss jetzt auch noch der Zugriff der Instanzvariablen in
\type{main} auf den Aufruf der Methode abgeändert werden:

\startcpp
cout << "Name: " << p.get_name() << ", Gewicht: "
     << p.get_weight() << endl;
cout << "Name: " << q.get_name() << ", Gewicht: "
     << q.get_weight() << endl;
\stopcpp

Da wir die Instanzvariablen mittels \type{private} verborgen haben,
müssen wir den Zugriff über die Methoden erfolgen lassen.

Jetzt macht es Sinn, das Schlüsselwort \type{struct} in \type{class}
umzuändern. Im Unterschied zu \type{struct}, ist die Bedeutung von
\type{class}, dass die Standardsichtbarkeit \type{private} ist. Das
zieht allerdings nach sich, dass die Sichtbarkeit der Methoden mittels
\type{public} sichergestellt werden muss, damit von \type{main} aus,
diese zugegriffen werden kann:

\startcpp
class Person {
    string name;
    double weight;
  public:
    Person(string name="", double weight=0) : name{name},
      weight{weight} { }
    string get_name() { return name; }
    double get_weight() { return weight; }
};
\stopcpp

Da die Standardsichtbarkeit von \type{class} eben \type{private} ist,
wurde die Deklaration der Instanzvariablen nach vor geschoben.
Allerdings ist es übersichtlicher, die alte Reihenfolge wieder
herzustellen, sodass die öffentliche Schnittstelle der Klasse wieder in
den Vordergrund rückt:

\startcpp
class Person {
  public:
    Person(string name="", double weight=0) : name{name},
      weight{weight} { }
    string get_name() { return name; }
    double get_weight() { return weight; }
  private:
    string name;
    double weight;
};
\stopcpp

Wir sehen hier, dass der öffentliche Teil der Klasse mit den Methoden am
Anfang steht und die Daten, die im Sinne der Kapselung von außen nicht
sichtbar und zugreifbar sein sollten, im privaten Teil am Ende
angeordnet sind. Damit muss zwar ein zusätzliches \type{private:}
angeführt werden, aber die Übersichtlichkeit konnte gesteigert werden.

Will man die Instanzvariablen verändern, dann werden Methoden notwendig,
die die Veränderung vornehmen. So eine Methode wird meist als
\quotation{setter method} bezeichnet:

\startcpp
void set_name(string name) { this->name = name; }
void set_weight(double weight) { if (weight > 0) this->weight = weight; }
\stopcpp

Als Alternative für diese Notation mittels \type{set_} und \type{get_}
bietet es sich auch an, auf überladene Methoden zurückzugreifen:

\startcpp
#include <iostream>

using namespace std;

class Person {
  public:
    Person(string name="", double weight=0) : name_{name}, weight_{weight} { }
    string name(string name="") { if (name != "") name_ = name; return name_; }
    double weight(double weight=0) { if (weight > 0) weight_ = weight; return weight_;}
  private:
    string name_;
    double weight_;
};

int main() {
    Person p{"Max", 79.5};
    Person q;
    cout << "Name: " << p.name() << ", Gewicht: " << p.weight() << endl;
    q.name("Mini");
    q.weight(60);
    cout << "Name: " << q.name() << ", Gewicht: " << q.weight() << endl;
}
\stopcpp

Als Vorteil lässt sich anführen, dass die Anzahl der Methoden reduziert
werden konnte und damit, aus einer gewissen Sichtweise, die Bedienung
der Klasse übersichtlicher wurde. Der Nachteil ergibt sich durch die
notwendige Abfrage im Rumpf der Methode.

\section[memberfunctions]{Methoden}

In diesem Abschnitt werden wir näher auf die verschiedenen Möglichkeiten
im Zusammenhang mit der Deklaration von Methoden eingehen. Methoden
werden in \cpp, wie schon erwähnt, als \quotation{member function}
bezeichnet.

\subsection[methoden-in-einer-.cpp-datei]{Methoden in einer \type{.cpp}
Datei}

Methoden können entweder direkt in einer Headerdatei definiert werden
(siehe Abschnitt \at[inlinemethods]) oder außerhalb in einer \type{.cpp}
Datei.

Hier betrachten wir den Fall, dass wir die Definition in einer
\type{.cpp} Datei vornehmen. In der Headerdatei befindet sich in der
Klasse lediglich die Deklaration der Methode:

\startcpp
// car.h
#ifndef CAR_H
#define CAR_H

#include <iostream>

class Car {
  public:
    void drive();
};
#endif
\stopcpp

Die Definition erfolgt in der zugehörigen Headerdatei:

\startcpp
// car.cpp
#include "car.h"

void Car::drive() {
    std::cout << "now driving..." << std::endl;
}
\stopcpp

Beachte, dass der Klassenname gemeinsam mit dem \type{::} Operator vor
den Methodennamen gesetzt wird, um dem Compiler mitzuteilen zu welcher
Klasse die Methode gehört.

Die Verwendung dieser Klasse findet typischerweise in einer anderen
Datei statt:

\startcpp
// methods.cpp
#include <iostream>

#include "car.h"

using namespace std;

int main() {
    Car car;
    car.drive();
}
\stopcpp

Bei der Übersetzung muss natürlich beachtet werden, dass das Modul
\type{car.cpp} zuerst übersetzt werden muss und danach entsprechend
gelinkt wird.

Im Abschnitt \at[classdeclaration] haben wir schon gesehen, dass
\type{this} einen Zeiger auf das aktuelle Objekt darstellt, der implizit
immer vorhanden ist, um auf die Instanzvariablen zuzugreifen, wenn sich
diese mit den Parametern überschneiden.

\type{this} kann allerdings auch als Rückgabewert verwendet werden:

\startcpp
// methodchaining.cpp
#include <iostream>

using namespace std;

class Adder {
  public:
    Adder& add(int val) { amount += val; return *this; }
    int value() { return amount; }
  private:
    int amount{};
};

int main() {
    Adder a;
    a.add(1).add(2).add(3);
    cout << a.value() << endl;
}
\stopcpp

Diese Art der Verarbeitung wird als Methodenverkettung (engl.
{\em method chaining}) zeichnet.

\subsection[inlinemethods]{\type{inline}-Methoden}

Wir haben schon gesehen, dass Methoden, die in der Klasse direkt
definiert werden, vom Compiler so betrachtet werden, als wenn man diese
mit \type{inline} gekennzeichnet hätte.

Wie auch normale Funktionen können Methoden explizit mit \type{inline}
gekennzeichnet werden. In diesem Fall muss die Definition der Methode
natürlich in der Headerdatei stehen und nicht in der \type{.cpp} Datei:

\startcpp
// inline_car.h
#ifndef INLINE_CAR_H
#define INLINE_CAR_H

#include <iostream>

class Car {
  public:
    void drive();
};

inline void Car::drive() {
    std::cout << "now driving..." << std::endl;
}

#endif
\stopcpp

Die Verwendung unterscheidet sich klarerweise nicht von der
nicht-\type{inline} Variante. Lediglich das zusätzliche Binden entfällt.

\subsection[constmethods]{\type{const}-Methoden}

Methoden können auch als \type{const} markiert werden, wobei die
Bedeutung darin liegt, dass auf konstante Objekte nur
\type{const}-Methoden, nicht jedoch nicht-\type{const}-Methoden,
aufgerufen werden können. Schauen wir uns das an Hand eines Beispiels
an, sodass die Anwendung einsichtig wird.

Ändere bitte die Definition von \type{value()} so ab, dass \type{const}
hinzugefügt wird:

\startcpp
int value() const { return amount; }
\stopcpp

Damit wird der Compiler die Datei wie gewohnt übersetzen und das
Ergebnis der Ausführung dieses Programmes wie vorher sein. Als nächsten
Schritt werden wir eine konstante Variable vom Typ \type{Adder} anlegen
und sonst wie vorher verfahren:

\startcpp
const Adder b;
b.add(1).add(2).add(3);
cout << b.value() << endl;
\stopcpp

Der Compiler wird diesen Programmtext jetzt mit einer Fehlermeldung
zurückweisen, da die Methode \type{add} nicht als \type{const} Methode
deklariert ist und hiermit es zu einer Veränderung einer als
\type{const} deklarierten Variable kommen würde.

Das Programm übersetzt nur, wenn die Methode \type{add()} für das Objekt
\type{b} entfernt wird, womit lediglich der Wert 0 ausgegeben wird.
Damit dieses Beispiel Sinn macht, muss man natürlich einen Konstruktor
hinzufügen:

\startcpp
Adder(int val=0) : amount{val} {}
\stopcpp

Mit einer Initialisierung der Variablen erhält man eine Konstante, die
in bestimmten Situationen sehr praktisch sein kann:

\startcpp
const Adder b{3};
cout << b.value() << endl;
\stopcpp

Es gibt Situationen in denen es angebracht ist, ein Objekt als
\type{const} zu markieren, aber gewisse Instanzvariablen sehr wohl
verändert werden sollen. Das werden wir im Abschnitt
\at[mutablemember] betrachten.

Die Vorteile solcher mit \type{const} markierten Methoden sind:

\startitemize[packed]
\item
  Der Compiler kann eine entsprechende Überprüfung vornehmen und damit
  sind unabsichtliche Veränderungen eines Objektes nicht mehr möglich.
\item
  Außerdem dient diese Markierung auch als Dokumentation gegenüber dem
  Benutzer der Klasse.
\stopitemize

\subsection[staticmethods]{\type{static}-Methoden}

\type{static}-Methoden (engl. {\em static method} oder {\em static
member function}) sind Methoden, die unabhängig von einer konkreten
Instanz einer Klasse aufgerufen werden können. Daher werden diese
Methoden auch Klassenmethoden genannt.

Das Entwurfsmuster \quotation{Factory method} behandelt das Anlegen
eines Objektes, dessen Typ im vorhinein nicht bekannt ist. Die Idee ist,
eine Methode zu schreiben, die aufgrund eines Parameters die richtige
Instanz anlegt und zurückliefert.

Nehmen wir an, dass wir eine Klasse \type{Car} schreiben wollen, die
eine allgemeine Beschreibung eines Autos darstellen soll. An konkreten
Automarken soll es \quotation{Andi} und \quotation{Zitrone} geben und in
Abhängigkeit einer Benutzereingabe ein Auto der entsprechenden Automarke
erzeugt werden:

\startcpp
// staticmembermethod.cpp
#include <iostream>

using namespace std;

class Car {
  public:
    static Car* make_car(string type);
};

class Andi : public Car {
};

class  : public Car {
};
\stopcpp

In diesem Beispiel sehen wir bis jetzt schon einige interessante
Aspekte:

\startitemize
\item
  In der Klasse \type{Car} gibt es eine statische Methode ({\em static
  member function}), die als Parameter den Autotypnamen erhält und einen
  Zeiger auf ein konkretes Autoobjekt zurückliefert.
\item
  Weiters gibt es die Klasse \type{Andi}, die von \type{Car} abgeleitet
  ist und keine weiteren Attribute oder Methoden definiert. Abgeleitete
  Klassen, also Vererbung, werden wir uns im Abschnitt
  \at[classinheritance] ansehen.
\stopitemize

Als nächstes werden wir die statische Methode \type{make_car}
implementieren:

\startcpp
Car* Car::make_car(string type) {
    if (type == "Audi")
        return new Audi();
    else if (type == "Skode")
        return new Skoda();
    else
        return nullptr;
}
\stopcpp

Diese Funktion ist von der Funktionsweise einfach und trotzdem sind
wiederum zwei Aspekte interessant:

\startitemize
\item
  Beachte, dass kein \type{static} in der Definition dieser Methode
  vorkommt.
\item
  Es wird das neue konkrete Auto am Heap angelegt. Wer die Verantwortung
  über dieses neu angelegte Objekt übernimmt, muss in der Dokumentation
  genau festgelegt werden.
\stopitemize

Die eigentliche Verwendung dieser Methode ist einfach:

\startcpp
int main() {
    string type;
    cout << "Autotype: ";
    cin >> type;
    Car* car{Car::make_car(type)};
}
\stopcpp

Hier siehst du, dass es kein Objekt der Klasse \type{Car} gibt und
trotzdem die Methode \type{make_car} aufgerufen werden kann, da diese
\type{static} deklariert ist. Beachte weiter, dass mittels des
Bereichsauflösungsoperators auf die Methode zugegriffen wird.

\section[datenmember]{Datenmember}

\subsection[staticmembervar]{\type{static}-Klassenvariablen}

Analog zu statischen Methoden ist es statische Variablen, die unabhängig
von bereits angelegten Objekten immer zur Verfügung stehen. Damit kann
sich die Klasse selbst einen Zustand abspeichern, der unabhängig von
einzelnen Instanzen ist. Solche Variablen werden auch als
Klassenvariablen bezeichnet.

Betrachten wir das nachfolgende Beispiel einer Klasse, die für jedes
neue Objekt eine neue fortlaufende Nummer, also so etwas wie eine
Seriennummer, vergibt:

\startcpp
#include <iostream>

using namespace std;

class Car {
  public:
    Car() : sernr_{maxsernr} { maxsernr++; }
    int sernr() { return sernr_; }
  private:
    static int maxsernr;
    int sernr_;
};

int Car::maxsernr{1};

int main() {
    Car c1;
    Car c2;
    Car c3;
    cout << "c1: " << c1.sernr() << endl;
    cout << "c2: " << c2.sernr() << endl;
    cout << "c3: " << c3.sernr() << endl;
}
\stopcpp

Die Ausgabe wird erwartungsgemäß folgendermaßen aussehen:

\startsh
c1: 1
c2: 2
c3: 3
\stopsh

Wichtig ist, dass \cpp es nicht erlaubt Klassenvariablen innerhalb der
Deklaration der Klasse vorzunehmen. Aus diesem Grund wurde diese, in dem
Beispiel, nach der Klassendeklaration vorgenommen.

\subsection[mutablemember]{\type{mutable}-Instanzvariablen}

Im Abschnitt \at[constmethods] haben wir schon konstante Methoden als
Hilfsmittel zur Sicherstellung der Unveränderbarkeit der Instanz
kennengelernt. Allerdings ist es manchmal sinnvoll, dass einzelne
Instanzvariablen von dieser Unveränderlichkeit ausgenommen sind.

Es gibt prinzipiell zwei Möglichkeiten wie diese Anforderung erfüllt
werden kann. Zuerst schauen wir uns eine Technik an, die sich
\quotation{Memoization} (das an das man sich erinnert) nennt. Verwendung
findet diese Technik überall wo aufwändige Berechnungen durchzuführen
sind oder bei dem Zugriff auf externe Ressourcen (Netzwerk, Dateien,
Datenbanken,\ldots{}).

Der Einfachheit halber betrachten wir eine Klasse \type{Polyline}, die
einen Polygonzug darstellen soll. Diese Klasse soll es ermöglichen eine
Folge von Punkten des zweidimensionalen Raumes als Polygonzug zu
interpretieren. An Operationen sind das Hinzufügen eines weiteren
Punktes und die Ermittlung der Länge interessant. Schauen wir uns dazu
einmal nur die Schnittstelle an:

\startcpp
// polyline.cpp
#include <iostream>
#include <vector>
#include <utility>
#include <cmath>

using namespace std;

class Polyline {
  public:
    Polyline(initializer_list<pair<double, double>> points);
    void add(double x, double y);
    double length() const;
};
\stopcpp

Wir sehen, dass vorerst alle implementierungsspezifischen Einzelheiten
weggelassen wurden.

Die Verwendung soll in weiterer Folge so funktionieren:

\startcpp
int main() {
    Polyline poly1{{}};
    cout << poly1.length() << endl;
    poly1.add(0, 0);
    cout << poly1.length() << endl;
    poly1.add(1, 1);
    cout << poly1.length() << endl;
    poly1.add(2, 0);
    cout << poly1.length() << endl;

    const Polyline poly2{{0, 0}, {1, 1}, {2, 0}};
    // poly2.add(3, 3);
    cout << poly2.length() << endl;
}
\stopcpp

Hier können wir schon sehen wie die Schnittstelle verwendet werden soll:

\startitemize[packed]
\item
  Man kann einen neuen Polygonzug mit einer Liste von Punkten anlegen.
\item
  Weiters man kann einen neuen Punkt hinten an den Polygonzug anhängen.
\item
  Außerdem kann man die Länge des Polygonzuges bestimmen.
\item
  Weiters kann ein Polygon konstant sein oder nicht. Wenn ein Polygonzug
  konstant ist, dann kann kein weiterer Punkt hinzugefügt werden.
\item
  Man kann dies zwar hieraus nicht ablesen, aber als Randbedingung soll
  gelten, dass die Länge eines Polygonzuges nur berechnet wird, wenn
  diese benötigt wird.
\stopitemize

Jetzt werden wir schrittweise auf die Implementierung eingehen. Starten
wir mit den nötigen Instanzvariablen:

\startcpp
private:
  vector<pair<double, double>> points;
  mutable double distance{-1};
\stopcpp

Offensichtlich wird die Liste der Punkte in einem Vektor gespeichert,
der Paare von \type{double}-Werten beinhaltet.

Weiters gibt es eine Instanzvariable \type{distance}, die die Länge
beinhaltet. Da die Länge nur berechnet werden soll, wenn diese benötigt
wird, wird in der Distanz ein ungültiger Wert eingetragen. Da eine
Distanz nie negativ sein kann, ist damit gekennzeichnet, dass der Wert
noch nicht ermittelt worden ist. Die Wirkungsweise von \type{mutable}
werden wir uns später noch ansehen.

Als nächstes werden wir uns die Implementierung des Konstruktors als
auch der Methode \type{add} ansehen:

\startcpp
Polyline(initializer_list<pair<double, double>> points) : points{points} {}
void add(double x, double y) { points.emplace_back(x, y); distance = -1; }
\stopcpp

Der Konstruktor ist wiederum sehr einfach, da lediglich der Vektor mit
den Punkten initialisiert wird. Auch die Implementierung der
\type{add}-Methode ist einfach. Neu ist lediglich die
\type{emplace_back} Methode, die ein neues Paar ohne Zwischenobjekt zum
Vektor hinzufügt. Damit wird markiert, dass die Distanz nicht mehr
gültig, wird der Wert auf einen ungültigen Wert zurückgesetzt.

Jetzt zur Implementierung der Methode \type{length}:

\startcpp
double length() const {
    if (distance >= 0)
        return distance;
    else {
        const pair<double, double>* p_prev{};
        for (auto& p : points) {
            if (!p_prev) {
                p_prev = &p;
                distance = 0;
                continue;
            }
            distance += sqrt(pow(p.first - p_prev->first, 2) +
                        pow(p.second - p_prev->second, 2));
            p_prev = &p;
        }
        return distance;
    }
}
\stopcpp

Wie funktioniert diese Methode:

\startitemize[packed]
\item
  Wenn ein gültiger Wert der Länge schon vorhanden ist, wird dieser
  zurückgeliefert.
\item
  Anderenfalls muss dieser berechnet werden. Dazu wird über alle Punkte
  iteriert und jeweils der Abstand zum Vorgänger ermittelt und zur Länge
  addiert. Dazu wird der Satz von Pythagoras angewandt.
\item
  Lediglich beim ersten Punkt ist kein vorheriger Punkt vorhanden, daher
  muss dieser Fall extra behandelt werden.
\stopitemize

Jetzt sollte das Beispiel wie gefordert funktionieren!

Als Alternative kann zur Verwendung des Schlüsselwort \type{mutable},
kann man auch das pimpl-Idiom (pointer to implementation) eingesetzt
werden. Die Idee ist, die veränderlichen oder besser die gesamte
Implementierung der Klasse in ein eigenes Modul auszulagern und in der
eigentlichen Klasse nur einen Pointer auf ein Objekt dieser
Implementierung zu speichern. Damit benötigt man einerseits kein
\type{mutable} mehr (zumindest in der ursprünglichen Klasse) und
andererseits ist in der Schnittstellenklasse keine Implementierung mehr
sichtbar!

Das könnte folgendermaßen aussehen:

\startcpp
// pimpl_polyline.cpp
// all necessary includes go here

class Polyline {
  public:
    Polyline(initializer_list<pair<double, double>> points) : impl{new PolylineImpl{points}} {}
    void add(double x, double y) { impl->add(x, y); }
    double length() const { return impl->length(); }
    ~Polyline() { delete impl; }
  private:
    PolylineImpl* impl;
};
\stopcpp

Die Klasse \type{PolylineImpl} wird jetzt in einem eigenem Modul
implementiert, das lediglich in kompilierter Form dem API mitgegeben
wird.

\section[constructor]{Konstruktor und Destruktor}

vorgeben und verbieten

Default-Konstruktor, allgemeine Konstruktoren, Typumwandlungskonstruktor
Destruktor, delegierender Konstruktor

explicit: Initialisierung mit und ohne \type{=}

\startitemize
\item
  Initialisierungslisten: Wie schon gesehen: \type{vector<int> nums(10)}
  vs. \mono{vector<int> nums\{10\}}: Die Form mit den geschwungenen
  Klammern erzeugt eine Initialisierungsliste.

  \startitemize
  \item
    Hat der Typ einen Konstruktor mit Initialisierungsliste des
    entsprechenden Typs, dann wird dieser verwendet.

    Der eingebaute Typ Array wird so betrachtet als hätte dieser einen
    Konstruktor mit Initialisierungsliste.
  \item
    Wenn der Typ einen Konstruktor mit entsprechender Anzahl und Typ von
    Parametern hat, dann wird dieser genommen. Unter Umständen werden
    noch implizite Konvertierungen durchgeführt.

    Auch bei eingebeuten Datentypen wird diese Syntax verwendet.
    Beachte, dass nur werterhaltende Konvertierungen vorgenommen werden
    (also keine narrowing Konverterierungen).
  \item
    Hat der Typ keinen Konstruktor, aber öffentlich zugängige
    Instanzvariable, dann werden diese in der Reihenfolge der
    Deklaration initialisiert.
  \stopitemize
\stopitemize

Ressourcen mittels RAII (siehe \in[raii]).

\section[membertypen]{Membertypen}

\section[aufzählungen]{Aufzählungen}

enum und enum class

\section[forward-deklarationen]{Forward-Deklarationen}

gegenseitige Abhängigkeit von Klassen

\stopcomponent
