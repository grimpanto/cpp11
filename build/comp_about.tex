
\startcomponent comp_about
\product prod_book

\chapter{Über dieses Buch}
\section[was-ist-die-zielsetzung-dieses-buches]{Was ist die Zielsetzung
dieses Buches?}

Das Ziel dieses Buches ist es, den Leser {\em schnell} in die
{\em moderne} Programmierung mittels \cpp einzuführen, sodass dieser
{\em produktiv} Programme entwickeln kann.

Da der Zweck dieses Buches ist, den Leser schnell in die Programmierung
von modernem \cpp einzuführen, werden die {\em wesentlichen} Teile von
\cpp
beschrieben. Das bedeutet lediglich, dass diejenigen Teile weggelassen
werden, die für die moderne Entwicklung mit \cpp nicht oder nur selten
benötigt werden.

Da \cpp eine Programmiersprache ist, deren Wurzeln in die 70er Jahre des
letzten Jahrhunderts zurückgehen und die permament weiterentwickelt
wird, hat die Sprache mittlerweile einen großen Umfang und eine hohe
Komplexität erreicht. Durch neue Elemente, die hauptsächlich im Jahr
2011 in die Programmiersprache als \cppXI eingeflossen sind, wurden
Möglichkeiten geschaffen {\em einfacher} zu programmieren, sodass viele
ältere syntaktischen Elemente nicht mehr benötigt werden. Auch können
Programmiertechniken, die früher essentiell gewesen sind, heute durch
andere, einfachere Konstrukte ersetzt werden.

Aus diesem Grund wird auch nicht auf die Maschinen-nahe Programmierung
eingegangen, die gänzlich andere Anforderungen hat als die
Programmierung von Entwicklung von Programmen für den Endbenutzer oder
für server-seitige Programme. In diesem Sinne behandelt dieses Buch
high-level \cpp!

Das Buch soll also als eine {\em gut lesbare} Anleitung zum Erlernen der
wesentlichen Teile von \cppXI dienen, sodass schnell Programme mittels
\cppXI erstellt werden können, die produktiv eingesetzt werden können.

\section[für-wen-ist-dieses-buch-geschrieben]{Für wen ist dieses Buch
geschrieben?}

Dieses Buch habe ich für Personen geschrieben, die schon
{\bf grundlegende} Programmierkenntnisse besitzen. Damit bezeichne ich
Personen, die in einer objektorientierten Programmiersprache einfache
Programme programmieren können.

Da es sich bei diesem Buch um {\em kein} Buch für Programmieranfänger
handelt, bitte ich Anfänger in der Programmierung zuerst ein anderes
Buch zur Hand zu nehmen, das die grundlegenden Begriffe, die Konzepte
der Programmierung und die tatsächliche Programmierung vermittelt. Dann
kommen Sie wieder zu diesem Buch zurück!

Es kommt es eigentlich nicht darauf an, in welcher Pogrammsprache die
Programmiererfahrungen vorhanden sind, wichtiger ist vielmehr, dass
Programme mit Ein- und Ausgabe, (lokalen und globalen) Variablen,
Datentypen, Verzweigungen, Schleifen, Funktionen, Exceptions, Klassen
mit Methoden und Vererbung erstellt werden können. Grundlegende Begriffe
wie z.B.~Algorithmus (eine Beschreibung des Lösungsweges eines
Problems), der Begriff eines Prozesses (gestartetes Programm) oder
Zahlensysteme sollten ebenfalls geläufig sein.

\section[wie-ist-dieses-buch-aufgebaut]{Wie ist dieses Buch aufgebaut?}

Das Buch ist in Kapitel gegliedert, die jeweils in bestimmte Themen
einführen:

\startdescription{Kapitel 1}
  Im ersten Kapitel wird Grundlegendes zu \cpp erklärt. Hier geht es
  darum ein Gefühl für die Art dieser Programmiersprache zu erhalten und
  auch ein wenig über die Bedeutung dieser verbreitenden
  Programmiersprache zu erahnen. Außerdem wird kurz angerissen wie die
  Infrastruktur von \cpp
  aussieht und wie man mit \cpp prinzipiell umgeht.
\stopdescription

\startdescription{Kapitel 2}
  Dieses Kapitel liefert einen Schnelleinstieg in die prozedurale
  Programmierung mit \cpp. Hier findet sich ein Überblick über den
  grundlegenden Aufbau eines einfachen \cpp Programmes. Dazu werden
  eingebaute Datentypen und auch einen Datentyp der Standardbibliothek
  vorgestellt. Auch die wichtigsten Kontrollstrukturen werden
  beschrieben und die erste eigene Funktion wird entwickelt.
\stopdescription

\startdescription{Kapitel 3}
  Hier werden die die eingebauten Datentypen im Detail besprochen,
  sodass diese im echten Einsatz verwendet werden können.
\stopdescription

\section[wie-kann-ich-das-buch-durcharbeiten]{Wie kann ich das Buch
durcharbeiten?}

Dieses Buch ist in einzelne Kapitel gegliedert, die so angelegt sind,
dass man diese hintereinander, möglichst vom Anfang bis zum Ende
durcharbeitet. Durcharbeitet deshalb, da dieses Buch viele Beispiele
enthält, die zum Verständnis beitragen sollen. Probiere die Beispiele
aus und teste die Programme! Das ist wichtig.

Es ist leider so, dass wenige kurze Teile dieses Buches so geschrieben
haben werden müssen, dass sie sich auf Inhalte beziehen, die erst ein
nachfolgenden Teile beschrieben sind. Dies habe ich dann jeweils auch
vermerkt. Das bedeutet, dass diese Teile beim ersten Durcharbeiten nicht
(vollständig) verstanden werden müssen. Diese Teile sind so verfasst,
dass beim Durcharbeiten der nachfolgenden Abschnitte darauf
zurückgegriffen werden kann.

Da ein Ziel dieses Buches ist, eine gut lesbare Anleitung zu sein, wird
auf Wiederholungen des Stoffes weitgehend verzichtet. Wenn eine
Information in einem Kapitel schon dargelegt worden ist, dann wird diese
in den weiteren Kapitel als bekannt vorausgesetzt. Damit kann man nicht
einfach ein Kapitel für sich alleine durcharbeiten, wenn man nicht über
die Vorinformationen verfügt.

Es geht um modernes \cpp! Daher werde ich nicht explizit darauf
hinweisen, dass ein Feature unter Umständen erst in \cppXI hinzugekommen
ist. Wir gehen davon aus, dass wir einen \cppXI kompatiblen Compiler
haben. Manche Features, die in \cppXIV hinzugekommen sind, erwähne und
erkläre ich auch. Diese sind dann allerdings klar gekennzeichnet, da die
Untertützung durch die Compiler unter Umständen noch nicht gegeben ist.

Als Referenz ist das Buch \cite[stroustrup2013] vom Erfinder von \cpp zu
empfehlen. Es behandelt weitgehend alle Aspekte der Sprache.

Benötigt man eine schnelle Referenz, dann rate ich dringend die
Online-Hilfe unter \cite[cppreference] aufzusuchen. Diese enthält
wichtige Fakten zur Sprache und eine vollständige Referenz zur
Standardbibliothek von \cpp!

\section[beispielprogramme]{Beispielprogramme}

Jedes Thema wird an Hand von Beispielprogrammen demonstriert. Der
praktische Ansatz steht im Vordergrund.

Für jedes Beispielprogramm wird ein Dateiname im Text (in der Kodierung
UTF-8) angegeben, also könnte eine Datei zum Beispiel \type{hallo.cpp}
heißen.

Diese Beispielprogramme werden in kleinen Schritten entwickelt, wobei
für jeden relevanten Schritt der Sourcecode in einer eigenen Datei (mit
fortlaufenden Nummer) auf der Website XXX zur Verfügung steht. Wenn das
Programm \type{hallo.cpp} heißt, dann gibt es unter Umständen weitere
Programme, die \type{hallo2.cpp}, \type{hallo3.cpp},\ldots{} heißen
können.

Diese Beispielprogramme stehen als Sourcecode unter
\useURL[url1][http://xxx.com]\from[url1] zur Verfügung und sind je
Kapitel zusammengefasst. Sie sollen eine Unterstützung bieten, wenn
einmal etwas nicht so funktioniert, wie man es sich erwartet, weil sich
Fehler eingeschlichen haben.

Die Ausgaben meiner Programme beziehen sich auf ein 32 Bit Linux System,
aber die Programme funktionieren auch unter 64 Bit Systemen als auch
unter Windows und Mac OSX.

\section[feedback]{Feedback}

Ich freue mich über jedes Feedback zu diesem Buch, das ich mit sehr viel
Engagement und sehr sorgfältig ausgearbeitet habe. Leider gilt für ein
Buch genau die gleiche Regel wie für jedes hinreichend großes Programm:
Es gibt immer noch einen Fehler!

Wenn Sie in diesem Buch noch Fehler finden, eine
Verbesserungsmöglichkeit sehen, einen Hinweis für Erweiterungen geben
können oder einfach nur Bemerkungen für mich parat haben, dann bitte ich
Sie diese mir per E-Mail an \type{guenter.kolousek@gmail.com} zu senden.

\section[anrede-und-geschlechtsneutralität]{Anrede und
Geschlechtsneutralität}

Prinzipiell verwende ich in diesem Buch von nun an das informelle
\quotation{Du} anstatt dem formelleren \quotation{Sie}. Dies hat nichts
mit mangelndem Respekt zu tun, den ich dir als Leser entgegenbringe,
sondern trägt dem Umstand Rechnung, dass wir Programmierer doch eine
Gemeinschaft sind und in dieser Gemeinschaft ist das deutsche
\quotation{Du} für die Kommunikation üblich.

In diesem Buch wende ich mich an Personen beiderlei Geschlechts, möchte
jedoch nicht in eine der üblichen \quotation{geschlechtsneutralen}
Schreibweisen verfallen. Der folgende Text gibt ein Beispiel für solch
eine Schreibweise an:

\startblockquote
Als Entwickler bzw. Entwicklerin bekommst du vom Auftraggeber bzw. der
Auftraggeberin oft nur unklare Angaben bezüglich der Angaben zu den
Kunden bzw. den Kundinnen und dessen bzw. deren Bedürfnissen.
\stopblockquote

Mir ist bewusst, dass es natürlich auch andere geschlechtsneutrale
Schreibweisen verwendet werden, die meiner Meinung nach auch nicht
sonderlich lesbarer sind bzw.~der deutschen Sprache widersprechen.

Schauen wir uns einerseits das Wort \quotation{Lehrling} (das
österreichische Wort für Azubi) an, das grammatikalisch gesehen männlich
ist und andererseits das Wort \quotation{Koryphäe}, das in der deutschen
Sprache weiblich ist. D.h.~es heißt \quotation{der Lehrling} und
\quotation{die Koryphäe} obwohl es natürlich Personen beiderlei
Geschlechts gibt auf die diese Begriffe zutreffen.

In diesem Sinne verwende verwende ich, dem einfacheren Lesefluss
zuliebe, die Form \quotation{der Programmierer} und bitte die Leserinnen
dieses Buches aus den angegebenen Gründen sich ebenfalls angesprochen zu
fühlen.

\section[wer-hat-maßgeblich-an-der-entstehung-dieses-buches-beigetragen]{Wer
hat maßgeblich an der Entstehung dieses Buches beigetragen?}

Hauptsächlich gebührt mein Dank meiner Frau Michaela, dich mich schon
über so viele Jahre in meinem gesamten Werdegang unterstützt hat. Auch
in diesem konkreten Fall des Schreibens an diesem Buch hat sie mich
wieder einmal mit Tat und moralischer Unterstützung zum Gelingen dieses
vorliegenden Werkes beigetragen. Dafür und auch für das nicht unwichtige
Korrekturlesen möchte ich mich bei ihr ganz herzlich bedanken.

Weiters möchte ich mich bei (XXX Schüler) bedanken\ldots{}

\stopcomponent
